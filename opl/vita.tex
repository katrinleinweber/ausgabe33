\textbf{Karin Aleksander}, Diplomphilosophin, Dr.~phil.,
Wissenschaftliche Bibliothekarin (Master of Library Science); seit 1990
Aufbau und Leiterin der Genderbibliothek am Zentrum für
transdisziplinäre Geschlechterstudien der Humboldt-Universität zu Berlin

\textbf{Christina Beckmann}, Diplom-Bibliothekarin (WB); seit 2011
One-Person Librarian in der Bibliothek des Leibniz-Zentrums Allgemeine
Sprachwissenschaft in Berlin-Mitte. Schwerpunkte ihrer Aufgaben sieht
sie in der Weiterentwicklung des Serviceangebots für die
Wissenschaftler*innen des Zentrums, die sie auch als
Open-Access-Ansprechpartnerin unterstützt.

\textbf{Ute Czerwinski} (60) -- Diplom-Bibliothekarin (FH) in der
Technischen Bibliothek der Bombardier Transportation GmbH Hennigsdorf
(und der Vorgängerbetriebe) seit 1983. Studium an der Fachschule für
wissenschaftliches Bibliothekswesen Leipzig (jetzt HTWK Leipzig) von
1976-1979, verheiratet, eine erwachsene Tochter. Corinna Haas, MA
Europäische Ethnologie und Allgemeine und Vergleichende
Literaturwissenschaft; M.A.~LIS Library and Information Science.
Wissenschaftliche Bibliothekarin am ICI Berlin Institute for Cultural
Inquiry seit dessen Gründung 2007.

\textbf{Jana Haase}, *1966, Dipl.-Bibl., MA Slawistik und Ethnologie,
seit 2003 zuständig für Bibliothek und Archiv des
Berufsausbildungszentrums Lette Verein Berlin

\textbf{Maximilian Hallmann} ist Fachangestellter für Medien und
Informationsdienste, Fachrichtung Bibliothek. Die Ausbildung absolvierte
er von 2011 bis 2014 in der Stadtbibliothek Berlin-Pankow. Seit November
2014 ist er für die Bibliothek des Landesarchivs Berlin zuständig.

\textbf{Claudia Loest}, Leiterin der Bibliothek im Museum für
Kommunikation Berlin seit 2008. Sie ist Diplom-Bibliothekarin (FH), seit
über 35 Jahren ``von der Pike auf'' tätig in Bibliotheken und hat
kürzlich das Master-Studium Library and Information Science am IBI
Berlin absolviert.

\textbf{Iris Schewe}, *1965, Diplom-Bibliothekarin; seit 1989 zuerst am
Berlin Museum, dann seit Stiftungsgründung am Stadtmuseum Berlin tätig.

\textbf{Pamela Schmidt}. M.A., hat Nordamerikastudien an der Freien
Universit‰t Berlin und Bibliotheks- und Informationswissenschaft an der
Humboldt-Universität zu Berlin studiert. Seit April 2016 führt sie die
Hochschulbibliothek der SRH Hochschule der populären Künste (hdpk) in
Berin.

\textbf{Katja Schöppe-Carstensen}, M. A. (*1976); Kunsthistorikerin,
arbeitet in der Bildung und Vermittlung in der Hegenbarth Sammlung
Berlin sowie im Brücke-Museum und in der Alten Nationalgalerie
