Der Beitrag thematisiert das innovative Tagungsformat eines Barcamps am
konkreten Beispiel der im Rahmen des Forschungsverbundes Marbach Weimar
Wolfenbüttel durchgeführten Veranstaltung ``Data and Demons: Von
Bestanddaten zu Services''. Ziel war es, Vertreter/innen der
verschiedenen Fachdisziplinen an Bibliotheken, Archiven und Museen
zusammenzubringen, die im Feld der Digital Humanities arbeiten. Das
kommunikative Format stellt Veranstalter/innen und Teilnehmer/innen vor
neue Herausforderungen, eignet sich jedoch dank der agilen
Agendaentwicklung dafür, rasch neue Perspektiven auf bestandsbezogene
Forschung an Bibliotheken zu entwickeln und Impulse für die Digital
Humanities zu generieren. Einen guten Einstieg bot die Keynote von Ralf
Stockmann in der Herzog August Bibliothek Wolfenbüttel mit dem Titel
``Der Zauberlehrling war nicht als Anleitung gemeint'' über Künstliche
Intelligenz in Kultur- und Wissenschaftseinrichtungen.

The article broaches the issue of the conference format of a Barcamp
using the event Data and Demons: From Collections' Data to Services as
example. The event, held by the Research Association Marbach Weimar
Wolfenbüttel, aims to bring together scientists of various disciplines
who work in the field of Digital Humanities at libraries, archives und
museums. The participatory format of the event presents both hosts and
participants with new challenges and opportunities. Due to the
self-organizing nature of a Barcamp, the participants develop new
perspectives on collection-based research in libraries and create
momentum for the Digital Humanities. The keynote by Ralf Stockmann, for
example, offered an excellent introduction to Artificial Intelligence in
scientific and cultural institutions.
