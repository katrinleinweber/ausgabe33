The Arctic inspires awe. This unique region of the world has been
studied in many ways by many different disciplines. The discipline of
librarianship can also add to its study. In this article, the authors, a
practicing Canadian librarian at Brock University in Ontario and an
Inuktitut student enrolled at the same university, offer a suggested
role for libraries to play in the ongoing study of the Arctic. They
explore and describe the role of libraries in supporting native Arctic
language education. Support for learning and preserving native Arctic
languages can be found in library collections, spaces and services. This
article looks at support of native speakers and other interested
language learners, support of language research, support of language
preservation, and support of new publishing opportunities that can be
provided by or through libraries. These language support examples come
from a document analysis that perused web sites, conference proceedings,
published scholarship in the form of books and articles, newspaper
sources, and personal background knowledge of the authors. Documents
were collected, categorized, and described. The language support
categories that emerged illustrate the many different ways that
libraries can engage in native Arctic language education support. In
offering this role, the authors hope to provide a means for librarians
to learn more about the Arctic as well as a way for libraries to
contribute to knowledge of the Arctic.

Die Arktis weckt Ehrfurcht. Diese einzigartige Region der Welt wurde in
vielerlei Hinsicht von vielen verschiedenen Disziplinen untersucht. Die
Bibliotheksforschung kann auch dazu beitragen, die Arktis zu erkunden.
In diesem Artikel schlagen die Autorinnen, eine praktizierende
kanadische Bibliothekarin an der Brock University in Ontario und eine
Inuktitut-Studentin, die an derselben Universität eingeschrieben ist,
Bibliotheken eine Rolle in einer laufenden Studie zur Arktis vor. Sie
erforschen und beschreiben die Rolle von Bibliotheken bei der
Unterstützung der arktischen Sprachausbildung. Unterstützung bei dem
Erlernen und Bewahren der einheimischen arktischen Sprachen findet durch
die Bestände, Räumlichkeiten und Dienstleistungen der Bibliotheken
statt. Dieser Artikel befasst sich mit der Unterstützung von
Muttersprachlern und anderen interessierten Sprachstudenten sowie mit
der Unterstützung von Sprachforschung, Sprachbewahrung und von neuen
Veröffentlichungsmöglichkeiten, die von oder über Bibliotheken
bereitgestellt werden können. Die Beispiele für die Sprachunterstützung
stammen aus einer Dokumentenanalyse, die sowohl Websites,
Konferenzberichte, akademische Veröffentlichungen in Form von Büchern
und Artikeln und Zeitungsquellen als auch das persönliche
Hintergrundwissen der Autoren erforschte. Die Dokumente wurden
gesammelt, kategorisiert und beschrieben. Die auftretenden Kategorien
zur Sprachunterstützung illustrieren die mannigfaltige Art und Weise,
wie sich Bibliotheken an der Unterstützung der Sprache der Einheimischen
in der Arktis beteiligen können. Mit dieser Rolle hoffen die Autorinnen,
dass Bibliothekarinnen und Bibliothekare mehr über die Arktis erfahren
können und dass Bibliotheken einen Beitrag zum Wissen über die Arktis
leisten können.
