\hypertarget{intro}{%
\subsection{Intro}\label{intro}}

Diese Ausgabe der LIBREAS. Library Ideas zielt darauf, etwas über den
Alltag in Bibliotheken zu erfahren. Nicht über die Innovationsprojekte
oder grossen strategischen Entscheidungen, sondern über das, was Tag für
Tag in den Bibliotheken passiert. Wir hatten in der bibliothekarischen
Literatur einen Bias gegenüber diesem Alltag festgestellt, der
vielleicht auch damit zu tun hat, dass ein Grossteil des Personals
andere Aufgaben hat oder den Eindruck vermittelt bekommt, dass das, was
sie tagtäglich erleben, nicht relevant für andere Bibliotheken wäre. Es
ist aber relevant. Wir wollten und wollen es wissen und zeigen. Deshalb
wählten wir\footnote{Auf Vorschlag von Leslie Kuo (\href{http://www.lesliekuo.com}{http://www.lesliekuo.com}).} unter anderem eine
niedrigschwellige Form für solche Beiträge: Eine Umfrage unter
Bibliotheken mit fünf Fragen, deren Antworten unser Meinung nach jeweils
mehr über diesen Alltag aussagen würden, als Analysen von
Strategiepapieren und Jahresberichten.

Selbstverständlich gilt: Auch eine solche Umfrage ersetzt nicht Besuche
in Bibliotheken vor Ort, ersetzt auch nicht die Erfahrung, die sich
ansammelt, wenn man tatsächlich Tag für Tag in einer Bibliothek
arbeitet. Aber Sie bieten einen Einblick in sehr unterschiedliche
Bibliotheken: Spezialbibliotheken, Öffentliche Bibliotheken,
Wissenschaftliche Bibliotheken, in kleine Bibliotheken und in grossen,
in solche aus kleinen Gemeinden und grossen Städten aus Deutschland,
Österreich und Brasilien.

In dieser Rubrik präsentieren wir diese Antworten. Wir danken allen
Teilnehmenden nochmals sehr herzlich.

\hypertarget{was-uns-auffiel}{%
\subsection{Was uns auffiel}\label{was-uns-auffiel}}

Wenig überraschend war, dass sich in den Bibliotheken, im Ganzen
genommen, gedruckte Medien und elektronische Geräte gemeinsam finden.
Die ganzen Diskurse darum, wie beide voneinander abzugrenzen sind,
scheinen immer unsinniger zu werden. Einige Kolleginnen und Kollegen
zeigten uns als ihren Hauptarbeitsplatz vor allem Rechner, aber Bücher
dominieren doch. Nebenher gibt es auch eine Anzahl verschiedener anderer
Medien und Dokumente (Leihscheine). Auffällig ist die weiterhin
vorhandene Trennung zwischen Arbeitsplätzen in direktem Kontakt mit
Nutzerinnen und Nutzern und Arbeitsplätzen in Büros und Magazinen ohne
diesen Kontakt.

Auffällig ist zudem, wie oft bei der Frage, was vermisst werden würde,
Kolleginnen und Kollegen genannt wurden. Offenbar ist die Verbindung zur
eigenen Bibliothek auch eine positive Verbindung mit den Personen, denen
dort tagtäglich begegnet wird.

Was fehlt? In den Bibliotheken fast durchgängig Platz, wobei nicht
einfach nach mehr Platz verlangt wird, sondern schon konkrete
Vorstellungen vorhanden sind, wie dieser genutzt werden sollte. Offenbar
planen Bibliothekarinnen und Bibliothekare -- nicht nur die Direktionen
-- immer schon eine mögliche Zukunft für ihre Einrichtungen, mit mehr
oder anderen Angeboten und einer Infrastruktur, die noch besser für ihre
Nutzerinnen und Nutzer funktioniert. Es geht ihnen nicht um ein reines
\enquote{einfach weiter so}.

Alle Bibliotheken haben etwas, was Sie als besonders ansehen. Jede
Bibliothek ist anders. Aufgefallen ist uns allerdings, wie oft in den
Bildern und Texten Zeitungen und Zeitschriften vorkamen.

Interessant (und ermutigend) fanden wir, wie oft in den Beiträgen auf
die eigene Sprache geachtet wird. In Zeiten wie diesen, in denen es sich
nicht selten so anfühlt, als würde es einen gesellschaftlichen
Rückschritt geben, zeigen die Beiträge aus den Bibliotheken eher eine
Gruppe, die daran interessiert ist, eine inklusive Sprache zu nutzen und
niemanden auszuschließen. Wir begrüßen das sehr.

Am Ende wollen wir die Bewertung unseren Leserinnen und Lesern
überlassen, gerade denen, die in Bibliotheken tätig sind: Finden Sie
sich und ihre Erfahrungen widergespiegelt? Oder ist sie ganz anders?
Sollten wir die Umfrage wiederholen, damit Sie auch Ihre Erfahrungen
darstellen können? -- Schreiben Sie uns:
\href{mailto:redaktion@libreas.eu}{\nolinkurl{redaktion@libreas.eu}}

Ihre / Eure Redaktion LIBREAS