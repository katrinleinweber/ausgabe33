\documentclass[a4paper,
fontsize=11pt,
%headings=small,
oneside,
numbers=noperiodatend,
parskip=half-,
bibliography=totoc,
final
]{scrartcl}

\usepackage{synttree}
\usepackage{graphicx}
\setkeys{Gin}{width=.4\textwidth} %default pics size

\graphicspath{{./plots/}}
\usepackage[ngerman]{babel}
\usepackage[T1]{fontenc}
%\usepackage{amsmath}
\usepackage[utf8x]{inputenc}
\usepackage [hyphens]{url}
\usepackage{booktabs} 
\usepackage[left=2.4cm,right=2.4cm,top=2.3cm,bottom=2cm,includeheadfoot]{geometry}
\usepackage{eurosym}
\usepackage{multirow}
\usepackage[ngerman]{varioref}
\setcapindent{1em}
\renewcommand{\labelitemi}{--}
\usepackage{paralist}
\usepackage{pdfpages}
\usepackage{lscape}
\usepackage{float}
\usepackage{acronym}
\usepackage{eurosym}
\usepackage[babel]{csquotes}
\usepackage{longtable,lscape}
\usepackage{mathpazo}
\usepackage[normalem]{ulem} %emphasize weiterhin kursiv
\usepackage[flushmargin,ragged]{footmisc} % left align footnote
\usepackage{ccicons} 

%%%% fancy LIBREAS URL color 
\usepackage{xcolor}
\definecolor{libreas}{RGB}{112,0,0}

\usepackage{listings}

\urlstyle{same}  % don't use monospace font for urls

\usepackage[fleqn]{amsmath}

%adjust fontsize for part

\usepackage{sectsty}
\partfont{\large}

%Das BibTeX-Zeichen mit \BibTeX setzen:
\def\symbol#1{\char #1\relax}
\def\bsl{{\tt\symbol{'134}}}
\def\BibTeX{{\rm B\kern-.05em{\sc i\kern-.025em b}\kern-.08em
    T\kern-.1667em\lower.7ex\hbox{E}\kern-.125emX}}

\usepackage{fancyhdr}
\fancyhf{}
\pagestyle{fancyplain}
\fancyhead[R]{\thepage}

% make sure bookmarks are created eventough sections are not numbered!
% uncommend if sections are numbered (bookmarks created by default)
\makeatletter
\renewcommand\@seccntformat[1]{}
\makeatother


\usepackage{hyperxmp}
\usepackage[colorlinks, linkcolor=black,citecolor=black, urlcolor=libreas,
breaklinks= true,bookmarks=true,bookmarksopen=true]{hyperref}
%URLs hart brechen
\makeatletter 
\g@addto@macro\UrlBreaks{ 
  \do\a\do\b\do\c\do\d\do\e\do\f\do\g\do\h\do\i\do\j 
  \do\k\do\l\do\m\do\n\do\o\do\p\do\q\do\r\do\s\do\t 
  \do\u\do\v\do\w\do\x\do\y\do\z\do\&\do\1\do\2\do\3 
  \do\4\do\5\do\6\do\7\do\8\do\9\do\0} 
% \def\do@url@hyp{\do\-} 
\makeatother 

%meta
%meta

\fancyhead[L]{B. Kaden \\ %author
LIBREAS. Library Ideas, 33 (2018). % journal, issue, volume.
\href{http://nbn-resolving.de/}
{}} % urn 
% recommended use
%\href{http://nbn-resolving.de/}{\color{black}{urn:nbn:de...}}
\fancyhead[R]{\thepage} %page number
\fancyfoot[L] {\ccLogo \ccAttribution\ \href{https://creativecommons.org/licenses/by/3.0/}{\color{black}Creative Commons BY 3.0}}  %licence
\fancyfoot[R] {ISSN: 1860-7950}

\title{\LARGE{Warum Forschungsdaten nicht publiziert werden}} % title
\author{Ben Kaden} % author

\setcounter{page}{1}

\hypersetup{%
      pdftitle={Warum Forschungsdaten nicht publiziert werden},
      pdfauthor={Ben Kaden},
      pdfcopyright={CC BY 3.0 Unported},
      pdfsubject={LIBREAS. Library Ideas, 33 (2018).},
      pdfkeywords={Open Data, Wissenschaftliches Publizieren, Praxis, Dissertation, Graue Literatur},
      pdflicenseurl={https://creativecommons.org/licenses/by/3.0/},
      pdfcontacturl={http://libreas.eu},
      baseurl={http://libreas.eu},
      pdflang={de},
      pdfmetalang={de}
     }



\date{}
\begin{document}

\maketitle
\thispagestyle{fancyplain} 

%abstracts

%body
{[}Vorbemerkung: Eine erste Version dieses Textes erschien am 13. März
2018 im LIBREAS. Library Ideas-Blog\footnote{\url{https://libreas.wordpress.com/2018/03/13/forschungsdatenpublikationen/}}.
Da das Thema der (Nicht-)Publikation von Forschungsdaten offensichtlich
eines der aktuellen Trendthemen im wissenschaftlichen Publikationswesen
ist, soll er im Rahmen der Rubrik \emph{LIBREAS. Dokumentation} in einer
durchgesehenen Form zweitveröffentlicht werden. Der Autor beschäftigt
sich aktuell im Rahmen des DFG-Projektes eDissPlus intensiv mit dem
Themenfeld der dissertationsbezogenen Forschungspublikation,was den
unmittelbaren Hintergrund dieses Textes bildet.{]}

Eine große und vermutlich noch zu wenig systematisierte Frage aller
Diskussionen um eine Offene Wissenschaft lautet zumindest für die in
diesem Bereich aktiven Infrastrukturen: Was spricht eigentlich dagegen?
Die Erfahrungen aus dem Open-Access-Bereich und mehr noch aus dem der
Open Science beziehungsweise Open Scholarship zeigen jedenfalls, dass es
nicht selten eine erhebliche Lücke zwischen Wünschen, Zielen und
Vorstellungen der Forschungsinfrastruktur und den besonders engagierten
fachwissenschaftlichen Vertreter*innen in diesem Bereich und einer
Gruppe gibt, die hier verkürzt als \enquote{Mainstream} der Wissenschaft
bezeichnet werden soll.

Eine wichtige, wenngleich auch nicht ganz überraschende Einsicht aus den
jahrelangen Auseinandersetzung mit der Offenen Wissenschaft muss lauten,
dass die meisten Forschenden vor allem forschen möchten und zwar in der
ihnen vertrauten Logik der wissenschaftlichen Publikationskulturen.
Defizite auch der Publikationssysteme werden durchaus erkannt, aber nur
dann tiefer adressiert, wenn sie zu spürbaren Behinderungen der
individuellen Forschung führen. In den meisten Fällen wollen Forschende
jedoch nicht als Innovator*innen für wissenschaftskommunikative und
-infrastrukturelle Lösungen in einer Weise aktiv werden, die zu einer
Umwidmung der Aufmerksamkeit vom Forschungsgegenstand auf diese
Metastrukturen der wissenschaftlichen Kommunikation führt. Wo also der
Leidensdruck im Umgang mit bestehenden Systemen und Praxen aus Sicht der
Forschenden nicht übermäßig hoch ist und tradierte Formen nach wie vor
die besten Karrierewege öffnen, werden auch hochengagierte und
raffiniert geschliffene Keynote-Adressen für eine Offfene Wissenscaft
wenig ändern. Für wissenschaftliche Bibliotheken und andere Akteure der
Wissenschaftsinfrastrukturen ist es folglich unerlässlich, zu wissen,
welche Ansprüche, Herausforderungen und Ziele in den einzelnen
Communities existieren. Dazu zählen auch die Gründe, warum
Forschungsdaten und -materialien disziplinär zwar unterschiedlich
intensiv aber nach wie vor eher insgesamt selten unter den
Idealvorstellungen der Offenen Wissenschaft zugänglich gemacht werden.

Auf dem am 12.März 2018 bei der Wikimedia durchgeführten
Open-Science-Bar-Camp\footnote{\url{http://www.open-science-conference.eu/barcamp/}}
des Leibniz Forschungsverbunds Science 2.0 gab es exakt dazu eine
Session mit dem Titel „Valid reasons for opting out of sharing openly``.
Einer Open-Science-Einstellung angemessen wurden einige Stichpunkte
freundlicherweise auch für alle die sichtbar, die nicht teilnehmen
konnten, in einem Etherpad hinterlegt.\footnote{\url{https://etherpad.wikimedia.org/p/oscibar2018_session13}}

Ich habe mir erlaubt, diese Stichpunkte zu clustern und
auszuformulieren. Im Anschluss an diese Liste ergänze ich noch einige
Stichpunkte aus dem eDissPlus-Projekt\footnote{\url{https://www2.hu-berlin.de/edissplus/}},
das sich mit den Möglichkeiten des dissertationsbegleitenden
Zugänglichmachens von Forschungsdaten befasste. Zu diesen zählen in der
Gesamtschau auch Daten, die man in weicheren Fächern eher als
Forschungsmaterialien bezeichnen würde, also beispielsweise Quellentexte
oder Bilder.

\textbf{Aufwand}

\begin{itemize}
\item
  Forschende wollen ihre Zeit lieber in die Forschung selbst als in die
  Organisation eines Austauschprozesses für Forschungsdaten investieren.
\item
  In der Projektplanung sind keine zeitlichen und personellen Ressourcen
  für die Aufbereitung von Forschungsdaten für ein Teilen
  beziehungsweise eine Veröffentlichung vorgesehen.
\item
  Die Veröffentlichung beziehungsweise Zugänglichmachung von
  Forschungsdaten wurden nicht bei der Projektplanung beziehungsweise
  beim Erstellen des Forschungsdatenmanagementplans berücksichtigt und
  ist nachträglich zu aufwändig umzusetzen.\footnote{Zum Thema
    Forschungsdatenmanagementpläne vergleiche auch ausführlicher: Kaden,
    Ben: Forschungsdatenmanagementpläne sind eine Grundbedingung guter
    Wissenschaft. Meint Nature. In: LIBREAS. Tumblr, 21.03.2018 Online:
    \url{http://libreas.tumblr.com/post/172103346561/}}
\end{itemize}

\textbf{Datenschutzrecht}

\begin{itemize}
\item
  Die Veröffentlichung beziehungsweise Zugänglichmachung von
  Forschungsdaten ist aus datenschutzrechtlichen Gründen ausgeschlossen.
\item
  Für eine Zugänglichmachung oder Publikation von personenbezogenen
  Daten liegt keine informierte Einwilligung vor.
\end{itemize}

\textbf{Institutionelle / infrastrukturelle Ausstattung}

\begin{itemize}
\tightlist
\item
  Die eigene Einrichtung bietet keine ausreichende Unterstützung sowohl
  infrastrukturell als auch beratend für die Verfügbarmachung
  beziehungsweise Publikation von Forschungsdaten an.
\end{itemize}

\textbf{Institutionelle Vorgaben}

\begin{itemize}
\item
  Prüfungsordnungen untersagen Promovierenden eine Publikation von
  Teilen der Promotion vor Abschluss des Promotionsverfahrens.
\item
  Es gibt keine formalen Auswahlkriterien, welche Forschungsdaten wie
  zugänglich gemacht werden sollten.
\end{itemize}

\newpage 

\textbf{Möglichkeiten und Kompetenzen des Teilens / Publizierens}

\begin{itemize}
\item
  Wissenschaftler*innen ist nicht bekannt, wo sie ihre Daten für eine
  Weitergabe hinterlegen können.
\item
  Wissenschaftler*innen sind nicht zureichend geschult, um
  Forschungsdaten wissenschaftlichen Publikationsstandards entsprechend
  zugänglich zu machen oder zu publizieren.
\item
  Forschungsdatenpublikationen sollen ein Peer-Review-Verfahren
  durchlaufen, das jedoch möglicherweise noch nicht existiert. Die nicht
  peer-reviewte Publikation von Forschungsdaten wird abgelehnt.
\end{itemize}

\textbf{Persönliche Einstellung / Datenkontrolle / Wissenschaftsethik}

\begin{itemize}
\item
  Wissenschaftler*innen sind am Thema Open Science / Offene Wissenschaft
  nicht interessiert.
\item
  Wissenschaftler*innen möchten gern wissen, wer ihre Forschungsdaten
  nachnutzt, weshalb sie diese nur auf persönliche Anfrage weitergeben
  würden beziehungsweise sich vorbehalten, eine Weitergabe abzulehnen.
\item
  Kooperationspartner in einem Forschungsdaten sprechen sich gegen eine
  Verfügbarmachung oder Publikation der im Projekt erzeugten
  Forschungsdaten aus.
\item
  Die Zugänglichmachung von Forschungsdaten wird bewusst verweigert,
  weil entsprechende Anregungen und Vorgaben als Eingriff in die
  persönliche Wissenschaftsfreiheit interpretiert werden.
\item
  Die eigenen Forschungsdaten werden als für eine Weitergabe zu wenig
  relevant eingeschätzt.
\item
  Wissenschaftler*innen möchten verhindern, dass ihre Forschungsdaten
  für von ihnen nicht gewünschte Zwecke nachgenutzt werden.
\item
  Es bestehen Zweifel daran, dass Dritte die Forschungsdaten bzw.
  Forschungsmaterialien wissenschaftlichen Standards entsprechend nutzen
  können.
\item
  Es besteht die Sorge, dass durch Zugänglichmachung von Forschungsdaten
  Schwächen der Datenerhebung und -analyse sichtbar werden.
\item
  Die konkreten Forschungsdaten sind in einer Weise manipuliert, die
  verborgen bleiben soll.
\end{itemize}

\textbf{Verlags-, Urheber- und Nutzungsrecht}

\begin{itemize}
\item
  Wissenschaftler*innen haben die Nutzungs- und Verwertungsrechte im
  Zuge einer Copyright-Vereinbarung an einen Wissenschaftsverlag
  übertragen und besitzen daher keine Verfügungsmöglichkeiten zum Teilen
  und Veröffentlichen von Forschungsdaten.
\item
  Promovierende, deren Forschungsprojekt in Kooperation mit Dritten
  stattfindet, haben nur begrenzt Verfügungsrechte über ihre
  Forschungsdaten. Dies betrifft insbesondere Kooperationen mit
  kommerziellen Partnern.
\item
  Es ist nicht bekannt, wer die rechtliche Eigentümerschaft zu den
  jeweiligen Forschungsdaten besitzt.
\end{itemize}

\textbf{Weitere Rechtsgebiete / Wissenschaftsethik}

\begin{itemize}
\item
  Das Forschungsthema ist zu sensibel als dass die Forschungsmaterialien
  und Forschungsdaten frei und international verfügbar gemacht werden
  können.
\item
  Es ist unklar, wer langfristig die Verantwortung für die jeweiligen
  Forschungsdaten und -materialien übernimmt.
\end{itemize}

\textbf{Wissenschaftssoziologie}

\begin{itemize}
\item
  Forschungsdaten und -materialien gelten als wissenschaftliches Kapital
  und werden (noch) zurückgehalten, weil sie in einem späteren Projekt
  weiter ausgewertet werden sollen.
\item
  Forschungsdaten und Forschungsmaterialien sollen als exklusives Asset
  für einen Antrag auf Projektförderung angeführt werden. Sind sie frei
  verfügbar, sinkt, so die Wahrnehmung, die Chance auf Förderung.
\item
  Forschungsdaten sollen zunächst exklusiv weiter beforscht werden,
  weshalb eine Publikation beziehungsweise Zugänglichmachung bestenfalls
  nach einem Embargo in Frage kommt.
\item
  Die Publikation oder das Teilen Forschungsdaten wird nicht ausreichend
  als wissenschaftliche Leistung gewürdigt.
\end{itemize}

\textbf{Wissenschaftsfreiheit}

\begin{itemize}
\tightlist
\item
  Das Prinzip der Open Science / Offenen Wissenschaft sollte nicht als
  Druck wirken -- im Sinne der Wissenschaftsfreiheit sollten
  Wissenschaftler*innen selbst entscheiden ob beziehungsweise wie und
  wann sie Forschungsdaten zugänglich machen.
\end{itemize}

Aus den Erfahrungen des eDissPlus-Projektes, das Einstellungsmuster von
Promovierenden zum Publizieren von Forschungsdaten untersuchte, lassen
sich, wie angekündigt, noch einige weitere Hürden benennen bzw. genannte
Aspekte weiter differenzieren. Dies sind unter anderem:

\textbf{Aufwand}

\begin{itemize}
\tightlist
\item
  Der Aufwand für eine dissertationsbegleitende
  Forschungsdatenpublikation wird, wie auch schon bei der
  Forschungsplanung erwähnt, nur sehr selten in der Dissertationsplanung
  und -- sofern überhaupt vorhanden -- in
  Forschungsdatenmanagementplänen berücksichtigt.
\end{itemize}

\textbf{Institutionelle und disziplinäre Vorgaben / Rahmenbedingungen}

\begin{itemize}
\item
  In vielen Bereichen fehlen für Forschungsdatenmanagement und das
  Publizieren von Forschungsdaten Standards, die eine Orientierung geben
  können.
\item
  Forschungsdatenpolicies werden im Einzelfall häufig als untauglich
  empfunden, unter anderem da sie zum Beispiel datenschutzrechtliche
  sowie weitere rechtliche Einschränkungen einer möglichen
  Forschungsdatenpublikation in keiner Weise würdigen.
\item
  In vielen Disziplinen gibt es keinen nachhaltigen und systematischen
  Austausch darüber, welchen Stellenwert und welche Form
  Forschungsdatenpublikationen für wissenschaftliche Kommunikation haben
  sollten.
\item
  Prüfungsordnungen treffen in der Regel keine Aussagen zu
  Forschungsdatenpublikationen und bieten daher auch keine Orientierung.
\item
  Für den Titelerwerb sind Forschungsdatenpublikationen in den meisten
  Fällen nicht erforderlich.
\end{itemize}

\textbf{Kompetenzen und Kompetenzvermittlung}

\begin{itemize}
\item
  Bereits für das generelle Forschungsdatenmanagement werden häufig
  Vermittlungsdefizite benannt: Das Thema findet in Lehre und
  Methodenausbildung kaum statt. Die Frage der
  Forschungsdatenpublikation wird in der Regel überhaupt nicht
  angesprochen.
\item
  Da in vielen Disziplinen Forschungsdatenpublikationen unüblich sind,
  kommen Promovierende auch bei der Literatursuche nicht mit dieser
  Gattung in Kontakt. Die Idee und Möglichkeit einer
  Forschungsdatenpublikationen ist ihnen daher häufig nicht bekannt.
\end{itemize}

\textbf{Persönliche Einstellung / Datenkontrolle / Wissenschaftsethik}

\begin{itemize}
\item
  Forschungsdaten werden im Rahmen von Promotionsprojekten häufig mehr
  als Mittel zum Zweck als als eigene publikationswürdige Größe
  angesehen.
\item
  Promovierende sehen sich angesichts der geringen Etablierung von
  Forschungsdatenpublikationen in vielen Bereichen überfordert und nicht
  in der Lage, zusätzlich zu ihrer Promotion entsprechende Pionierarbeit
  für die Publikationskulturen ihrer Fächer zu leisten.
\end{itemize}

\textbf{Rechtliche Aspekte}

\begin{itemize}
\item
  Promovierende können oft die urheber- beziehungsweise
  erheber-rechtlichen Folgen einer Forschungsdatenpublikation nur
  unzureichend abschätzen. So scheint die Bedeutung der
  Creative-Commons-Lizenzen für konkrete Szenarien oft wenig eindeutig.
\item
  Das Datenschutzrecht steht einer Publikation von
  dissertationsbegleitenden Forschungsdaten mit Personenbezug aktuell in
  fast allen Fällen im Weg, unter anderem da selten
  Forschungsdatenpublikationen von vornherein eingeplant und in den
  jeweiligen informierten Einwilligungen nicht vorkommen. Das
  nachträgliche Einholen eine Publikationserlaubnis ist oft nicht
  möglich oder wird als deutlich zu aufwändig eingeschätzt.
\end{itemize}

\textbf{Wissenschaftssoziologie / Forschungsdatenkontrolle}

\begin{itemize}
\item
  Forschungsdatenpublikationen versprechen in den meisten Fällen keinen
  zusätzlichen Reputationsgewinn. Teilweise wird von den Gutachter*innen
  eine dissertationsbegleitende Forschungsdatenpublikation sogar als
  potentiell schädlich eingeschätzt.
\item
  Forschungsdaten gelten bei vielen Promovierenden als
  wissenschaftliches Kapital. Besteht die Bereitschaft zur Weitergabe,
  wird eine selektive Zugänglichmachung auf Anfrage deutlich gegenüber
  einer allgemeinen Zugänglichmachung als Publikation bevorzugt.
\end{itemize}

Beide Auflistungen sind sicher keinesfalls erschöpfend. Zu vielen
Aspekten wären auch vertiefende Einzeluntersuchungen sinnvoll und
notwendig. Deutlich wird jedoch bereits an dieser losen Reihung, dass
individuelle Einstellungsmuster zwar einen wichtigen Aspekt darstellen
und entsprechend Lobbyarbeit für Open Access und weitere Elemente der
Offenen Wissenschaft sicher sinnvoll ist. Nachhaltig wirksam werden sie
aber nur sein können, wenn auch entsprechende Rahmenbedingungen
existieren und zwar sowohl infrastrukturell als auch fachkulturell.

Ein offensichtliches Haupthindernis liegt sicher im aktuell in vielen
Fällen deutlichen Missverhältnis von Aufwand und Nutzen. Eine
wissenschaftlichen Standards entsprechende Forschungsdatenpublikation
erfordert gerade angesichts des Mangels an Best-Practice-Beispielen und
auch im Einzelfall passenden Leitlinien eine vergleichsweise hohe
zusätzliche Arbeitsbelastung, der jedoch kein erwartbarer
Reputationsgewinn entgegensteht. Je niedrigschwelliger hier
Infrastrukturen Beratung und andere Dienste anbieten können, desto
besser. Die Universitätsbibliothek wurde von vielen der befragten
Promovierenden im eDissPlus-Projekt als natürliche Ansprechpartnerin für
alle Fragen zum Thema Forschungsdaten angesehen und zwar auch für
Aspekte, die man gemeinhin eher den Instituten und der dortigen
Ausbildung zugeschrieben hätte. Man wünscht sich von der Bibliothek
idealerweise ein umfassendes Spektrum an Dienstleistungen von der
Beratung über Cloud-Dienste bis zur Langzeitarchivierung für komplexe
Datenstrukturen. Was davon wie tatsächlich angeboten werden kann, ist
allerdings eine andere Diskussion. In der Erwartungshaltung der
Promovierenden, die mit hoher Kompetenz an Forschungsdaten gehen, ist
Github ein Benchmark. Bei den anderen eher Dropbox. Im Ergebnis weist
der Wunsch in eine Richtung, die beide Dienste mit umfassenden
Beratungsangeboten verbindet und überschätzt nebenbei deutlich die
Entwicklungskapazitäten, die die öffentliche Hand an dieser Stelle
bereitzustellen vermag.

Die zweite große Herausforderung liegt in einer unklaren Rechtslage in
Bezug auf Forschungsdaten. Hierzu gab es im Januar 2018 einen Workshop
an der Viadrina in Frankfurt/Oder,\footnote{\url{http://www.forschungsdaten.org/index.php/Rechtliche_Aspekte_bei_digitalen_Forschungsdaten}}
der erwartungsgemäß wenige Antworten dafür aber noch tiefere Einblicke
in die Komplexität der Gemengelage bot.

Und schließlich fehlen für viele Disziplinen tatsächlich praktikable
Infrastrukturangebote, auch übrigens von Verlagen oder anderen
kommerzieller aufgestellten Anbietern, für eine zeitgemäße und
dauerhafte Zugänglichmachung von Forschungsdatenpublikationen. Das
Druckparadigma, dass sich im PDF-Format vergleichsweise angenehm
spiegeln ließ, funktioniert für digitale Forschungsdaten endgültig nicht
mehr. Will man sie zum festen Teil der wissenschaftlichen Kommunikation
werden lassen, benötigt man oftmals überhaupt erst einmal adäquate
mediale Präsentationsformen -- eine Debatte übrigens, die in zahlreichen
Bereichen bestenfalls nebenbei geführt wird. Implizit lässt sich hier
auch aus den eDissPlus-Befragungen ein weiteres sehr großes Hindernis
für Forschungsdatenpublikationen ermitteln: Die Daten sind unter
Umständen mit den bestehenden Möglichkeiten gar nicht sinnvoll als
Publikation darstellbar.

Zu all diesen Problemen existiert selbstverständlich engagierte Arbeit
hinsichtlich möglicher Lösungen, auch wenn die Digitalisierung der
Wissenschaft gerade im Infrastrukturbereich noch ganz anders auch von
den Träger- und Förderinstitutionen adressiert werden könnte, als dies
bislang geschieht. So ermöglicht beispielsweise der edoc-Server seit
diesem Jahr Forschungsdatenpublikationen der Humboldt-Universität zu
Berlin.\footnote{\url{https://www.hu-berlin.de/de/pr/nachrichten/maerz-2018/nr_180309_00}}
Auch Zenodo kann als gelungenes Beispiel für einen zeitgemäßen
Publikationsserver für eine Vielzahl denkbarer Materialien gelten. Dass
Forschungsdaten auf den Publikationsservern mit Metadaten erschlossen,
mit DOIs versehen werden und wenigstens teilweise sogar in
Bibliothekskatalogen bibliografiert erscheinen, mag ebenfalls ein früher
Schritt in Richtung Anerkennung als ordentliche wissenschaftliche
Publikation sein. Aber damit endet in den meisten Fällen die Reichweite
dessen, was Bibliotheken und Infrastrukturen zu leisten in der Lage
sind.

Die Selbstorganisation der Wissenschaft macht es erforderlich, dass sich
die Fachkulturen darüber verständigen, welchen Stellenwert in welcher
Form die Publikation von Forschungsdaten und anderen
Forschungsmaterialien für sie einnehmen kann und soll. Sie müssen selbst
ausdiskutieren, prüfen und entscheiden, ob sie zum Beispiel ein Peer
Review wollen, ob komplexe Forschungsdatenpublikationen auch
berufungsrelevant sein können, welche Formate sie bevorzugen und welche
Metadaten sie brauchen. Die Infrastrukturseite kann aufzeigen, was
möglich ist, kann Erfahrungen, Erkenntnisse und Überblickswissen
vermitteln. Dafür brauchen wir Veranstaltungen wie das
Open-Science-Bar-Camp und Wissenschaftsforschung, wie sie im
eDissPlus-Projekt stattfinden konnte. Die Absicherung eines Wissenstands
auf der jeweiligen Höhe der Zeit zu den Praxen und Wünschen der
Fachkulturen einerseits und den technischen Möglichkeiten andererseits
ist bereits für sich eine enorme Herausforderung und zugleich
Minimalbedingung jeder zielorientierten Infrastrukturentwicklung.
Bereits dafür benötigt man, wenn man so will, Brückenakteure, die sowohl
Fach- und Publikationskulturen als auch Ziele, Möglichkeiten, Grenzen
von Wissenschaftsinfrastruktur und -organisation kennen. Man braucht
solche Akteure aber noch mehr, wenn es darum geht, den eigentlichen
Schritt einer digitalen Wissenschaft zu gehen, nämlich die Infrastruktur
mit der wissenschaftlichen Kommunikation und an bestimmten Stellen
direkt mit der Forschung zu verzahnen. Wir können auf Barcamps und in
Workshops umfassend darüber diskutieren, warum Forschende ihre Daten
nicht publizieren. Greifbare und praktikable Lösungen werden sich jedoch
erst dann daraus ableiten lassen, wenn diese Diskussionen auch mit den
Wissenschaftler*innen geführt werden. Dazu ist es notwendig, beide
Seiten nicht nur zu kennen, sondern in einem stetigen Dialog zu halten.
Ich habe eingangs bemerkt, dass Forschende vor allem forschen und sich
möglichst wenig mit Infrastrukturfragen befassen wollen. Dies ändert
sich bei der digitalen Wissenschaft natürlich dann, wenn Infrastruktur
und Forschung zusammenfallen. Ein gutes Beispiel unter anderem auch für
die Schwierigkeiten dieser Entwicklung sind die Digital Humanities.

Wir, als Vertreter zum Beispiel der Universitätsbibliotheken, bemühen
uns unter anderem in Projekten wie eDissPlus intensiv darum, zu
verstehen, was die Forschenden als Zielgruppen umtreibt. Konsequent
gedacht könnte sich das Konzept der Zielgruppe allerdings an nicht
wenigen Stellen zunehmend relativieren und das Gewicht deutlich in
Richtung einer Partnerschaft verschieben. Ein unmittelbares Desiderat
ist aktuell ein Forum oder eine Form, das beziehungsweise die es uns
ermöglicht, Erkenntnisse wie die oben zusammengetragenen in einen
übergreifenden und gestaltungsorientierten Dialog mit allen Stakeholdern
einzubringen. Ein zweites ist häufig eine stabile und ein idealerweise
unkomplizierte Struktur, die es nach Ende von Projekten von eDissPlus
erlaubt, über die, wenn man so will, Anamnese hinaus, zu konkretisieren,
wie, mit welchen Mitteln und an welchen Stellen die in diesem Fall
identifizierten Hürden abgebaut werden können. Diese Situation steht
dabei exemplarisch für etwas sehr Generelles: Die Ansprüche einer
Offenen Digitalen Wissenschaft werden sich nur als Projekt des
Gesamtsystems Wissenschaft realisieren lassen.

%autor
\begin{center}\rule{0.5\linewidth}{\linethickness}\end{center}

\textbf{Ben Kaden} arbeit an der Universitätsbibliothek der
Humboldt-Universität zu Berlin und ist Mitherausgeber der LIBREAS.

\end{document}
