\documentclass[a4paper,
fontsize=11pt,
%headings=small,
oneside,
numbers=noperiodatend,
parskip=half-,
bibliography=totoc,
final
]{scrartcl}

\usepackage{synttree}
\usepackage{graphicx}
\setkeys{Gin}{width=.4\textwidth} %default pics size

\graphicspath{{./plots/}}
\usepackage[ngerman]{babel}
\usepackage[T1]{fontenc}
%\usepackage{amsmath}
\usepackage[utf8x]{inputenc}
\usepackage [hyphens]{url}
\usepackage{booktabs} 
\usepackage[left=2.4cm,right=2.4cm,top=2.3cm,bottom=2cm,includeheadfoot]{geometry}
\usepackage{eurosym}
\usepackage{multirow}
\usepackage[ngerman]{varioref}
\setcapindent{1em}
\renewcommand{\labelitemi}{--}
\usepackage{paralist}
\usepackage{pdfpages}
\usepackage{lscape}
\usepackage{float}
\usepackage{acronym}
\usepackage{eurosym}
\usepackage[babel]{csquotes}
\usepackage{longtable,lscape}
\usepackage{mathpazo}
\usepackage[normalem]{ulem} %emphasize weiterhin kursiv
\usepackage[flushmargin,ragged]{footmisc} % left align footnote
\usepackage{ccicons} 

%%%% fancy LIBREAS URL color 
\usepackage{xcolor}
\definecolor{libreas}{RGB}{112,0,0}

\usepackage{listings}

\urlstyle{same}  % don't use monospace font for urls

\usepackage[fleqn]{amsmath}

%adjust fontsize for part

\usepackage{sectsty}
\partfont{\large}

%Das BibTeX-Zeichen mit \BibTeX setzen:
\def\symbol#1{\char #1\relax}
\def\bsl{{\tt\symbol{'134}}}
\def\BibTeX{{\rm B\kern-.05em{\sc i\kern-.025em b}\kern-.08em
    T\kern-.1667em\lower.7ex\hbox{E}\kern-.125emX}}

\usepackage{fancyhdr}
\fancyhf{}
\pagestyle{fancyplain}
\fancyhead[R]{\thepage}

% make sure bookmarks are created eventough sections are not numbered!
% uncommend if sections are numbered (bookmarks created by default)
\makeatletter
\renewcommand\@seccntformat[1]{}
\makeatother


\usepackage{hyperxmp}
\usepackage[colorlinks, linkcolor=black,citecolor=black, urlcolor=libreas,
breaklinks= true,bookmarks=true,bookmarksopen=true]{hyperref}
%URLs hart brechen
\makeatletter 
\g@addto@macro\UrlBreaks{ 
  \do\a\do\b\do\c\do\d\do\e\do\f\do\g\do\h\do\i\do\j 
  \do\k\do\l\do\m\do\n\do\o\do\p\do\q\do\r\do\s\do\t 
  \do\u\do\v\do\w\do\x\do\y\do\z\do\&\do\1\do\2\do\3 
  \do\4\do\5\do\6\do\7\do\8\do\9\do\0} 
% \def\do@url@hyp{\do\-} 
\makeatother 

%meta
%meta

\fancyhead[L]{S. Hotze \\ %author
LIBREAS. Library Ideas, 33 (2018). % journal, issue, volume.
\href{http://nbn-resolving.de/}
{}} % urn 
% recommended use
%\href{http://nbn-resolving.de/}{\color{black}{urn:nbn:de...}}
\fancyhead[R]{\thepage} %page number
\fancyfoot[L] {\ccLogo \ccAttribution\ \href{https://creativecommons.org/licenses/by/3.0/}{\color{black}Creative Commons BY 3.0}}  %licence
\fancyfoot[R] {ISSN: 1860-7950}

\title{\LARGE{Menschen -- Bücher -- Wissenssuche \\
Eine Woche in der UB Erlangen-Nürnberg (20. November – 24. November 2017)}} % title
\author{Steffi Hotze} % author

\setcounter{page}{1}

\hypersetup{%
      pdftitle={Menschen -- Bücher -- Wissenssuche},
      pdfauthor={Steffi Hotze},
      pdfcopyright={CC BY 3.0 Unported},
      pdfsubject={LIBREAS. Library Ideas, 33 (2018).},
      pdfkeywords={Wissenschaftliche Bibliothek, Berufspraxis, Universitätsbibliothek Nürnberg-Erlangen},
      pdflicenseurl={https://creativecommons.org/licenses/by/3.0/},
      pdfcontacturl={http://libreas.eu},
      baseurl={http://libreas.eu},
      pdflang={de},
      pdfmetalang={de}
     }



\date{}
\begin{document}

\maketitle
\thispagestyle{fancyplain} 

%abstracts

%body
Die Universitätsbibliothek Erlangen-Nürnberg (UB) der
Friedrich-Alexander-Universität (FAU) besteht aus vier großen
Zweigbibliotheken (wovon die größte die Hauptbibliothek ist) und 130
Teilbibliotheken.

Die größte Teilbibliothek ist die Jurabibliothek. Sie hat einen Bestand
von circa 250.000 Bänden (Gesamtbestand UB: 5.6 Millionen Bände),
aufgestellt nach der Regensburger Verbundklassifikation (RVK), und 300
Arbeitsplätze für Bibliotheksnutzer\_innen. Da sie eine
Präsenzbibliothek ist, erstrecken sich die Öffnungszeiten über die
gesamte Woche; Montag bis Samstag von 8.00--23.45 Uhr und Sonntag von
12.00--19.45 Uhr.

\hypertarget{menschen}{%
\section*{Menschen}\label{menschen}}

Das Stammpersonal der Jurabibliothek umfasst drei Personen. Unsere
Leiterin, die ebenfalls die Fachreferentin für Recht ist und zwei
Diplom-Bibliothekare\_innen in Vollzeit, das sind mein Kollege und ich.

Daneben gibt es, wie in vielen anderen Universitätsbibliotheken auch,
eine Vielzahl studentischer Hilfskräfte. Bei uns in der
Bibliotheksverwaltung arbeiten drei Hilfskräfte, die verschiedene
Arbeiten übernehmen wie zum Beispiel das Bekleben und Stempeln von
Büchern, das Einlegen von Loseblattnachlieferungen oder die Bearbeitung
von FAUdok-Bestellungen, dem internen Dokumentenlieferdienst. ~\\
In der Bibliothek arbeiten zwölf Hilfskräfte, welche die Bücher in die
Regale räumen und allgemein für Ordnung in den Regalen sorgen. Um die
Öffnungszeiten zu gewährleisten, arbeiten 14 Hilfskräfte an der Pforte
der Bibliothek.

Am Anfang der Woche wird immer der Besucherzähler abgelesen. Für die
Kalenderwoche 46 sind~3986 Besuche gezählt worden. Der Durchschnitt
liegt bei circa 3500. In Zeiten von Haus- und/oder Seminararbeiten
steigt die Zahl auch auf 5000--6000 Besuche in der Woche.

\hypertarget{buxfccher}{%
\section*{Bücher}\label{buxfccher}}

Die Zeit vor der Öffnung der Bibliothek nutze ich mehrmals die Woche um
einen Kontrollgang zu machen. Dabei räume ich die Bücher, die auf den
Tischen zurückgelassen wurden, auf die Bücherwagen. Beim Gang durch die
Regalreihen stelle ich die Bücher wieder gerade hin, kontrolliere, ob
sie richtig eingeordnet sind, stelle Stellvertreter zurück und schaue,
ob Bücher in den Regalen versteckt wurden.~Nebenbei müssen auch
Kleinigkeiten erledigt werden, wie Stühle an die Tische stellen,
vergessene Körbe aufräumen, Tischlampen ausschalten.

Seit diesem Jahr sind wir nicht nur für die Einarbeitung der Bücher im
Lokalsystem verantwortlich, sondern auch für den gesamten
Erwerbungsprozess. Wir konnten somit erfolgreich den integrierten
Geschäftsgang einführen. Wir verwalten die Studienbeitragsmittel für
Literatur des Fachbereichs Jura und die Lehrstuhlmittel einiger
Lehrstühle (nicht jeder wollte die Erwerbung an uns abgeben, das war den
Lehrstühlen freigestellt).

Die Erwerbungsentscheidung wird weiterhin durch die Lehrstühle
getroffen, um alle weiteren Vorgänge kümmern wir uns. Da mein Kollege
und ich beide vorher in der Hauptbibliothek in der Erwerbungsabteilung
tätig waren, konnten wir schnell eine gute Arbeitsroutine einführen.~~

Die Bestellungen erreichen uns überwiegend per E-Mail oder als
Zettelsammlung. Darüber hinaus haben die Lehrstühle Daueraufträge bei
unserem lokalen Buchhändler angelegt, so werden viele der neu
erschienenen Lehrbücher und Kommentare automatisch geliefert.

Das Bearbeiten der Bücher macht einen großen Teil unserer Arbeit aus.
Bücher werden täglich geliefert -- unsere Buchhändler aus Erlangen und
Nürnberg bringen sie persönlich vorbei, alles andere kommt per Post. Das
heißt auch, am Anfang der Arbeit steht immer das Auspacken. Dann erfolgt
ein erstes Vorsortieren: Was wird aus welchem Etat bezahlt, was aus
Studienbeiträgen oder vom Lehrstuhl-Etat? Zuerst bearbeiten wir die
Bücher für die Lehrstühle, da die Professoren\_innen oder
Lehrstuhl-Mitarbeiter\_innen die Bücher meist kurzfristig für ihre
Forschungen benötigen.~Die Masse der Bücher sind allerdings die, die bei
uns in der Bibliothek stehen werden.~

Am schnellsten zu bearbeiten sind Neuauflagen von Lehrbüchern und
Kommentaren. Die Signatur ist schon vorhanden und es muss lediglich die
Auflagenzählung geändert werden. Die weiteren Arbeitsschritte sind vor
allem Fleißarbeit -- Mediennummer ins Buch kleben, Signatur ins Buch
schreiben und die Bücher im Erwerbungssytem erfassen.

Da wir nur begrenzten Platz haben, stehen bei uns nur die aktuelle und
die Vorauflage in voller Stückzahl im Regal. Ältere Auflagen werden bis
auf ein Exemplar ausgesondert und meist an die Studierenden verschenkt.

Wir katalogisieren mit Aleph im Bayerischen Bibliotheksverbund. Durch
den großen Anteil deutscher Literatur sind viele Katalogisate bereits
auf Stufe 5 katalogisiert, das heißt die Autopsie wurde von
Kollegen\_innen anderer Bibliotheken bereits durchgeführt. Im Idealfall
wurde schon eine RVK- Signatur ermittelt. Wenn für ein Buch mehrere
Systematikstellen in Frage kommen, bevorzugen wir die Fachsystematik „P"
(P steht in der RVK für Rechtswissenschaft).

Die Bezahlung der Rechnungen erfolgt direkt über den Erwerbungsclient
und die Haushaltsschnittstelle. Am nächsten Morgen bekommen wir ein
Protokoll per E-Mail zugeschickt, mit dem die Rechnungen nochmals
abgeglichen werden. Wenn Rechnungen und Protokoll übereinstimmen, wird
das Geld an die Lieferanten überwiesen.

Durch die intensive Nutzung der Bestände haben wir häufig Probleme mit
defekten Büchern, überwiegend mit gebrochenen Buchblöcken oder
abgerissenen Buchrücken. Wir versuchen diese Schäden so gut wie möglich
selbst zu reparieren. Wenn die Bücher nicht zu reparieren sind, werden
sie ausgesondert und teilweise auch neu gekauft.

Die Teilbibliothek ist nicht an den deutschen Leihverkehr angeschlossen,
hin und wieder sind Ausnahmen möglich. Durch einen unserer Professoren
haben wir einen Spezialbestand zum Islamischen Recht, darunter
Primärquellen. Wir haben daher häufig den Alleinbesitz für Deutschland,
daher geben wir, in Absprache mit dem Professor, die Bücher in die
Fernleihe.

Auf der Suche nach Büchern kommen die Studierenden vor allem dann zu
uns, wenn sie das Buch zwar im Online-Katalog, aber nicht in der
Bibliothek gefunden haben. Leider sind verschwundene Bücher tatsächlich
ein Problem. Wobei wir aus Erfahrung sagen können, dass es sich zum
großen Teil nicht um Diebstahl handelt.

Meistens suchen wir selber nochmal nach dem Buch, ich kann aber nicht
sagen wie hoch unsere Erfolgsquote ist. Bei verstellten Büchern sind wir
recht gut beim Aufspüren. Wir haben Erfahrungswerte, wohin ein Buch mit
einer bestimmten Signatur am wahrscheinlichsten verstellt wurde.

Bücher, die schon lange verschwunden sind, können wir nachkaufen. Wir
schreiben zudem Rundmails an die Lehrstühle, wenn wir auf der Suche nach
verschwundenen Büchern sind. Manchmal sind die Bücher aber auch sehr gut
versteckt und man kann sich sicher sein, dass sie nach dem Ende der
Hausarbeiten-Zeit sofort wieder auftauchen.~

Unsere fünf häufigsten Ursachen für Bücher, die nicht an ihrem Platz im
Regal stehen:

\begin{enumerate}
\def\labelenumi{\arabic{enumi}.}
\item
  Jemand anders nutzt das Buch gerade beziehungsweise es liegt auf einem
  Bücherwagen zum Zurückstellen (sehr wahrscheinlich, wenn anmStelle des
  Buches nur eine Lücke gefunden wurde).
\item
  Jemand vom Lehrstuhl hat es mitgenommen, aber keinen Stellvertreter
  ausgefüllt.
\item
  Jemand vom Lehrstuhl hat es mitgenommen, einen Stellvertreter
  ausgefüllt, aber der steht nicht mehr an seinem Platz. (Leider allen
  sie immer wieder aus den Regalen).
\item
  Das Buch wurde nicht an der richtigen Stelle einsortiert.
\item
  Das Buch wurde versteckt.
\end{enumerate}

\hypertarget{wissenssuche}{%
\section*{Wissenssuche}\label{wissenssuche}}

Einmal in der Woche habe ich an der Information (kurz: Info) in der
Hauptbibliothek Dienst. Die Info ist immer mit zwei Personen besetzt.
Sie ist die zentrale Anlaufstelle für alle Nutzer\_innen. Wir helfen bei
der Recherche, bei der Nutzung der technischen Geräte und beim
Zurechtfinden in der Bibliothek. Wie viele andere Bibliotheken können
auch wir den Trend beobachten, dass viele Anfragen zur Technik im Haus
kommen. Vor allem zur Nutzung der Internetarbeitsplätze, der
Multifunktionskopierer, des Buchscanners oder Fragen zum W-LAN.

Daneben gibt es viele Fragen zur Nutzung von E-Books und E-Journals: Da
jeder Anbieter andere Modalitäten hat, treten hier häufig
Nutzungsprobleme auf. Ein weiteres Problem ist für viele Nutzer\_innen,
ein Buch oder eine Zeitschrift zu finden, wenn ihre Quelle eher spärlich
oder nicht aussagekräftig ist.

Ich arbeite auch im Schulungsteam in der Hauptbibliothek mit. In dieser
Woche durfte ich zwei Schulungen halten, an der sechs beziehungsweise
neun Studierende teilgenommen haben -- pro Schulung können maximal 15
Personen teilnehmen. Bei unseren „Basisschulungen" werden Grundlagen zur
Nutzung unseres Bibliothekssystems vermittelt. Wir erklären den Aufbau
unseres Bibliothekssystems und die verschiedenen Möglichkeiten der
Recherche, danach bekommen die Teilnehmer\_innen Aufgabenblätter und
üben direkt im Online-Katalog die Suche nach Büchern, Zeitschriften und
Aufsätzen.

Auch in der Jurabibliothek wird geschult: Im Wintersemester 2017/2018
haben wir,~inklusive unserer Leiterin, erstmals Bibliotheksschulungen
für die Studierenden im ersten Semester Rechtswissenschaft durchgeführt.
Früher wurde dies immer von den Tutoren\_innen des Fachbereichs
übernommen. Wir hoffen, dass durch unseren eigenen Einsatz die
Studierenden den Online-Katalog intensiver nutzen und die Bibliothek und
auch wir als Mitarbeiter\_innen gleich etwas vertrauter wirken.

Wir planen für Ende des Semesters eine weiterführende Schulung, in der
wir vor allem auf die Nutzung der wichtigsten juristischen Datenbanken
eingehen. Außerdem wollen wir beispielhaft zeigen, wie die Studierenden
FAUdok und die Fernleihe nutzen können. Dieses Schulungskonzept richtet
sich dann nicht nur an Erstsemester, sondern an alle Interessierten.

%autor

\end{document}
