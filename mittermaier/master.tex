\documentclass[a4paper,
fontsize=11pt,
%headings=small,
oneside,
numbers=noperiodatend,
parskip=half-,
bibliography=totoc,
final
]{scrartcl}

\usepackage{synttree}
\usepackage{graphicx}
\setkeys{Gin}{width=.4\textwidth} %default pics size

\graphicspath{{./plots/}}
\usepackage[ngerman]{babel}
\usepackage[T1]{fontenc}
%\usepackage{amsmath}
\usepackage[utf8x]{inputenc}
\usepackage [hyphens]{url}
\usepackage{booktabs} 
\usepackage[left=2.4cm,right=2.4cm,top=2.3cm,bottom=2cm,includeheadfoot]{geometry}
\usepackage{eurosym}
\usepackage{multirow}
\usepackage[ngerman]{varioref}
\setcapindent{1em}
\renewcommand{\labelitemi}{--}
\usepackage{paralist}
\usepackage{pdfpages}
\usepackage{lscape}
\usepackage{float}
\usepackage{acronym}
\usepackage{eurosym}
\usepackage[babel]{csquotes}
\usepackage{longtable,lscape}
\usepackage{mathpazo}
\usepackage[normalem]{ulem} %emphasize weiterhin kursiv
\usepackage[flushmargin,ragged]{footmisc} % left align footnote
\usepackage{ccicons} 

%%%% fancy LIBREAS URL color 
\usepackage{xcolor}
\definecolor{libreas}{RGB}{112,0,0}

\usepackage{listings}

\urlstyle{same}  % don't use monospace font for urls

\usepackage[fleqn]{amsmath}

%adjust fontsize for part

\usepackage{sectsty}
\partfont{\large}

%Das BibTeX-Zeichen mit \BibTeX setzen:
\def\symbol#1{\char #1\relax}
\def\bsl{{\tt\symbol{'134}}}
\def\BibTeX{{\rm B\kern-.05em{\sc i\kern-.025em b}\kern-.08em
    T\kern-.1667em\lower.7ex\hbox{E}\kern-.125emX}}

\usepackage{fancyhdr}
\fancyhf{}
\pagestyle{fancyplain}
\fancyhead[R]{\thepage}

% make sure bookmarks are created eventough sections are not numbered!
% uncommend if sections are numbered (bookmarks created by default)
\makeatletter
\renewcommand\@seccntformat[1]{}
\makeatother


\usepackage{hyperxmp}
\usepackage[colorlinks, linkcolor=black,citecolor=black, urlcolor=libreas,
breaklinks= true,bookmarks=true,bookmarksopen=true]{hyperref}
%URLs hart brechen
\makeatletter 
\g@addto@macro\UrlBreaks{ 
  \do\a\do\b\do\c\do\d\do\e\do\f\do\g\do\h\do\i\do\j 
  \do\k\do\l\do\m\do\n\do\o\do\p\do\q\do\r\do\s\do\t 
  \do\u\do\v\do\w\do\x\do\y\do\z\do\&\do\1\do\2\do\3 
  \do\4\do\5\do\6\do\7\do\8\do\9\do\0} 
% \def\do@url@hyp{\do\-} 
\makeatother 

%meta
%meta

\fancyhead[L]{B. Mittermaier, C. Holzke, C. Frick, I. Barbers \\ %author
LIBREAS. Library Ideas, 33 (2018). % journal, issue, volume.
\href{http://nbn-resolving.de/}
{}} % urn 
% recommended use
%\href{http://nbn-resolving.de/}{\color{black}{urn:nbn:de...}}
\fancyhead[R]{\thepage} %page number
\fancyfoot[L] {\ccLogo \ccAttribution\ \href{https://creativecommons.org/licenses/by/3.0/}{\color{black}Creative Commons BY 3.0}}  %licence
\fancyfoot[R] {ISSN: 1860-7950}

\title{\LARGE{Open Access löst nicht alle Probleme – aber mehr, als mancher meint\\
Eine Replik zu: \enquote{Die Transformation des Publikationssystems zu Open Access und die Konsequenzen für Bibliotheken und Wissenschaft: Ausgewählte Aspekte.}}} % title
\author{Bernhard Mittermaier, Christoph Holzke, Claudia Frick, Irene Barbers} % author

\setcounter{page}{1}

\hypersetup{%
      pdftitle={Open Access löst nicht alle Probleme – aber mehr, als mancher meint. Eine Replik zu: Die Transformation des Publikationssystems zu Open Access und die Konsequenzen für Bibliotheken und Wissenschaft: Ausgewählte Aspekte.'},
      pdfauthor={Bernhard Mittermaier, Christoph Holzke, Claudia Frick, Irene Barbers},
      pdfcopyright={CC BY 3.0 Unported},
      pdfsubject={LIBREAS. Library Ideas, 33 (2018).},
      pdfkeywords={Open Access, Bibliothek, Wissenschaftskommunikation, Open-Access-Transformation, wissenschaftliche Publikationen, Transformationsverträge, Literaturversorgung},
      pdflicenseurl={https://creativecommons.org/licenses/by/3.0/},
      pdfcontacturl={http://libreas.eu},
      baseurl={http://libreas.eu},
      pdflang={de},
      pdfmetalang={de}
     }



\date{}
\begin{document}

\maketitle
\thispagestyle{fancyplain} 

%abstracts

%body
Rafael Ball setzt sich in einem Fachbeitrag in BIT online\footnote{Ball,
  Rafael (2018): Die Transformation des Publikationssystems zu Open
  Access und die Konsequenzen für Bibliotheken und Wissenschaft:
  Ausgewählte Aspekte. B.I.T. online 21(1), 9--17. Online verfügbar
  unter
  \url{http://www.b-i-t-online.de/heft/2018-01-fachbeitrag-ball.pdf}.
  Alle nicht anderweitig belegten Zitate im vorliegenden Artikel stammen
  aus dieser Publikation. In leicht gekürzter Fassung wurde der Beitrag
  außerdem in Forschung\&Lehre 3/2018 publiziert. Alle Internetquellen
  wurden am 07. März 2018 geprüft.} kritisch mit der Transformation des
Publikationssystems hin zu Open Access auseinander. Er gibt darin
\enquote{\emph{einen kurzen Überblick über ausgewählte, bislang wenig
beachtete und diskutierte Argumente}}, die aus seiner Sicht gegen Open
Access und insbesondere Transformationsverträge sprechen. Die AutorInnen
des hier vorliegenden Textes sehen in diesem Fachbeitrag eine ganze
Reihe Missverständnisse, Falschinterpretationen und innere Widersprüche.
Besonders verstörend wirkt der zweifelhafte Umgang mit Quellen. Um vorab
nur ein Beispiel zu nennen:

\begin{quote}
\emph{\enquote{Wenn der Wissenschaftler oder die Wissenschaftlerin das
Publikationsorgan nicht mehr auswählen darf, das er oder sie für
fachlich geeignet hält, sondern nur aus der Liste derjenigen
Zeitschriftentitel auswählen darf, für die eine APC-Flatrate vereinbart
wurde, wird das gewiss als Einschränkung der Wissenschaftsfreiheit
interpretiert und empfunden werden und teilweise auch faktisch so sein.
Erste gerichtliche Klagen sind dazu bereits in Deutschland anhängig.}}
\end{quote}

Als Beleg führt Ball einen Link zu einem Beitrag des Deutschlandfunks
an, der sich mit der Klage von Konstanzer Jura-ProfessorInnen gegen die
kostenfreie Artikel-Zweitnutzung befasst. Diese Klage (Singular!) wendet
sich gegen die \enquote{Satzung zur Ausübung des wissenschaftlichen
Zweitveröffentlichungsrechtes} der Uni Konstanz und folglich gegen einen
(vermeintlichen) Zwang zu Green Open Access. Sie hat hingegen nichts mit
Gold Open Access und Transformationsprozessen zu tun.

Die AutorInnen folgen gern der Anregung von Rafael Ball, seine Argumente
zu diskutieren. Dies erfolgt in thematischer Anordnung.

\hypertarget{die-perspektive-der-wissenschaftlerinnen}{%
\subsubsection{Die Perspektive der
WissenschaftlerInnen}\label{die-perspektive-der-wissenschaftlerinnen}}

Die wohl relevantesten Änderungen für WissenschaftlerInnen durch die
Transformation zu Open Access sind zum einen der barrierefreie Zugang zu
Fachliteratur und zum anderen die wegfallende Notwendigkeit, sich bei
der Nachnutzung mit Fragen des Copyrights zu befassen. Die
deutschlandweiten Initiativen, wie durch die Deutsche
Forschungsgemeinschaft (DFG) finanzierte Publikationsfonds und das
Projekt DEAL,\footnote{Siehe hierzu auch ein Interview über die
  DEAL-Verhandlungen in der vorherigen LIBREAS-Ausgabe: Bernhard
  Mittermaier (2017): Aus dem DEAL-Maschinenraum --- ein Gespräch mit
  Bernhard Mittermaier``. LIBREAS. Library Ideas, 32 (2017).
  \url{http://libreas.eu/ausgabe32/mittermaier/}} haben dies zum Ziel.
Ball suggeriert jedoch auf mehr oder weniger subtile Weise noch andere
angebliche Ziele, die ein \emph{\enquote{trauriges Resultat}} oder gar
eine \emph{\enquote{Einschränkung der Freiheit von Forschung und
Wissenschaft}} zur Folge haben sollen. Diese Ziele sind jedoch lediglich
angedichtet und so fällt die Argumentation von Ball bei einem Vergleich
mit der Realität schnell in sich zusammen.

Bereits das verwendete Wording ist irreführend und legt die Tendenz und
die Absicht des Textes offen. So wird beispielsweise als Gegensatz zum
derzeitigen \emph{\enquote{subskriptionsbasierten Modell}} der Begriff
\emph{\enquote{Author-Pays-Modell}} verwendet. Nur an einer Stelle wird
der tatsächlich gegensätzliche Begriff
\emph{\enquote{publikationsbasiertes Modell}} genannt. Wenn Ball schon
den einzelnen Wissenschaftler und die einzelne Wissenschaftlerin als
leidtragenden \enquote{Bezahler} inszenieren will, so sollte
konsequenterweise auch von einem \enquote{Reader-Pays-Modell} beim
\emph{subskriptionsbasierten Modell} gesprochen werden. Aber eigentlich
steht es außer Frage, dass ebenso wenig wie WissenschaftlerInnen heute
für das Lesen einzelner Artikel selber zahlen, diese nach der
Transformation zu Open Access selbst für das Publizieren aufkommen
müssen. Ganz im Gegenteil beweisen gerade die großen Initiativen wie
DEAL oder alternative Modelle wie arXiv und SCOAP,³ dass es auch künftig
die Bibliotheken sein sollen und werden, die hier für die
WissenschaftlerInnen die Publikationskosten tragen. Somit entfällt auch
das Argument, dass \emph{\enquote{eine Verschiebung von rein fachlichen
Argumenten hin zu einer wirtschaftlichen Entscheidung}} bei der Wahl des
Publikationsorgans durch WissenschaftlerInnen zu erwarten sei.

\begin{quote}
\emph{\enquote{Autoren können dann selbst entscheiden, in welchem
Journal sie publizieren wollen und welche Kosten sie für die
Veröffentlichung bereit sind zu zahlen. Die auf den ersten Blick
bestechende (und nicht falsche) Idee bedeutet allerdings gleichzeitig
eine Einschränkung der Forschungs- und Wissenschaftsfreiheit bei der
Wahl des Publikationsorgans und eine Verschiebung von rein fachlichen
Argumenten hin zu einer wirtschaftlichen Entscheidung. Dies scheint der
Freiheit von Forschung und Lehre zu widersprechen, führt sie doch von
einer Qualitätsentscheidung zu einer monetären Entscheidung, die zudem
nur den sehr speziellen Blick eines einzelnen Autors widerspiegeln
kann.}}
\end{quote}

Die gleich an mehreren Stellen als Sorge präsentierte Unterstellung, ein
publikationsbasiertes Modell wäre gleichbedeutend mit einer
Einschränkung der Wissenschaftsfreiheit oder würde diese in Gefahr
bringen, ist schlicht nicht haltbar. Zum Thema Open-Access-Publizieren
und Wissenschaftsfreiheit sah sich schon vor zehn Jahren die Allianz der
deutschen Wissenschaftsorganisationen mit einer derartigen
\emph{\enquote{inakzeptablen Unterstellung konfrontiert}} und wies diese
begründet von sich.\footnote{Allianz der deutschen
  Wissenschaftsorganisationen (2009): Open Access und Urheberrecht: kein
  Eingriff in die Publikationsfreiheit: Gemeinsame Erklärung der
  Wissenschaftsorganisationen.
  \url{http://gfzpublic.gfz-potsdam.de/pubman/item/escidoc:2875912}}
WissenschaftlerInnen publizieren die Ergebnisse ihrer Forschung, um eine
möglichst breite Rezeption zu erreichen und damit andere
WissenschaftlerInnen ihre Forschung darauf aufbauen können. Dem steht
das Subskriptionsmodell mit einem prinzipiell limitierten Zugang
fundamental entgegen. So haben WissenschaftlerInnen nicht nur rein
wirtschaftlich gesehen kein gesteigertes Interesse daran, dass ihre
Publikationen von Verlagen möglichst gewinnbringend verkauft werden,
sondern diese Praxis steht der Rezeption ihrer Forschungsergebnisse
sogar noch im Weg. Dies ist, entgegen dem von Ball vorgetragenen
\emph{\enquote{Kostenbewusstsein beim Autor}}, ein tatsächlich
relevantes Argument für Open Access. Noch einmal das schon genannte
Zitat in einem anderen Zusammenhang:

\begin{quote}
\emph{\enquote{Wenn der Wissenschaftler oder die Wissenschaftlerin das
Publikationsorgan nicht mehr auswählen darf, das er oder sie für
fachlich geeignet hält, sondern nur aus der Liste derjenigen
Zeitschriftentitel auswählen darf, für die eine APC-Flatrate vereinbart
wurde, wird das gewiss als Einschränkung der Wissenschaftsfreiheit
interpretiert und empfunden werden und teilweise auch faktisch so
sein.}}
\end{quote}

Die von Ball immer wieder vorgetragene Behauptung, WissenschaftlerInnen
dürften künftig nur noch bei Verlagen mit Transformationsverträgen
publizieren, ist schlichtweg falsch. Keine Initiative oder
Forschungseinrichtung hat ein derartiges Ziel. Ganz im Gegenteil kommen
die Publikationsfonds schon heute für alle Article Processing Charges
(APCs) für Publikationen in Gold-Open-Access-Zeitschriften auf,
unabhängig von der Existenz eines Rahmen- oder Transformationsvertrages.
Eine Änderung dieser Praxis ist nicht angekündigt. Diese Behauptung wird
auch durch das repetitive Vortragen im Text nicht wahrer. Genauso gut
könnte man als Gegner des Subskriptionsmodells behaupten,
WissenschaftlerInnen dürften derzeit nur Artikel in Zeitschriften lesen,
die die Bibliothek lizenziert hat, die Bibliothek würde für darüber
hinausgehende Literaturwünsche nicht aufkommen und dies wäre eine
Einschränkung der Wissenschaftsfreiheit. Auf diese kreative Idee ist
bisher nur niemand gekommen.

Diese Darstellung einer düsteren Zukunft auf Basis einer falschen
Annahme leitet auch direkt zum nächsten großen von Ball aufgespannten
Szenario über, nämlich der indirekten Einschränkung der
Wissenschaftsfreiheit durch fehlende Mittel aufgrund der angeblich nicht
vorhandenen Planbarkeit des Publikationsoutputs der
WissenschaftlerInnen. Ball malt dabei ein klares Bild.

\begin{quote}
\emph{\enquote{Wenn eine Hochschule oder eine Institution keine Mittel
mehr für die Veröffentlichung zur Verfügung hat oder zur Verfügung
stellen kann, können die Wissenschaftler eben nicht publizieren.}}
\end{quote}

An dieser Stelle sollte man erstmal einen Schritt zurücktreten und sich
klar werden, was diese Aussage impliziert. Zum einen wird behauptet,
dass das Publikationsaufkommen einer Forschungseinrichtung gänzlich
nicht planbar wäre und zum anderen, dass Bibliotheken mit einer gewissen
Unwägbarkeit in der Finanzplanung überfordert wären. Keine der beiden
Implikationen trifft zu. Auf die Finanzplanung von Seiten der Bibliothek
wird im Abschnitt \enquote{Aufgaben der Bibliotheken} näher eingegangen.
An dieser Stelle soll der Fokus auf dem Publikationsaufkommen liegen.

Das Publikationsaufkommen einer Forschungseinrichtung ist in der Tat
nicht konstant -- meist steigt dieses oder schwankt von Jahr zu Jahr.
Beispielsweise lagen die Schwankungen des Publikationsaufkommens von
Zeitschriftenartikeln der ETH Zürich und des Forschungszentrums Jülich
seit 2008 jährlich zwischen minus 8\,\% und plus 19\,\%.\footnote{Recherchiert
  auf ETH Zürich Research Collection
  \url{https://www.research-collection.ethz.ch} und Publikationsportal
  JuSER des Forschungszentrums Jülich \url{https://juser.fz-juelich.de}
  am 5. März 2018.} Das sind durchaus größere Variationen, aber von
einer gar nicht vorhandenen Planbarkeit kann dennoch nicht gesprochen
werden. Auch ist die Aussage, dass die \emph{\enquote{nicht
kalkulierbare Dauer des Peer Review-Verfahrens bei den Verlagen und die
damit zusammenhängende Annahme oder Ablehnung eines Manuskripts
{[}\ldots{}{]} eine mittelfristige Budgetplanung für zu erwartende
Publikationskosten, also die Bereitstellung der APCs, praktisch
unmöglich {[}machen{]}}} ein Scheinargument, wenn man von
kontinuierlichen Einreichungen durch WissenschaftlerInnen bei Verlagen
ausgeht. Nun könnte man anführen, dass Prognosen immer nur seriöse
Versuche sind, in die Zukunft zu schauen und dadurch immer in Gefahr
nicht einzutreffen. Diese Art der Planung ist für Bibliotheken aber
nichts Neues und bei einer vorausschauenden bibliothekarischen Planung,
wie sie im Abschnitt \enquote{Aufgaben der Bibliotheken} vorgestellt
wird, sollten WissenschaftlerInnen genauso frei überall und bis zum
Jahresende publizieren können, wie sie bei der jetzigen Planung Artikel
lesen können.

Ein weiteres Chaos-Szenario, das Ball für die WissenschaftlerInnen
zeichnet, beinhaltet einen bürokratischen Mehraufwand für eben diese,
der ihnen angeblich von den Bibliotheken zugeschoben wird, sollten im
Rahmen der Big Deals keine Flatrates vereinbart werden. Er spricht
davon, dass \emph{\enquote{Tausende einzelner Autoren ihre Verträge über
ihre Publikationen und die zu zahlenden APCs jeweils individuell
abschließen}} müssten. In der Tat wäre es furchtbar, müssten
WissenschaftlerInnen für jede ihrer Publikationen einen einzelnen
Vertrag mit dem Verlag schließen, dessen Inhalt oft nicht allgemein
verständlich ist. Dann hätte man nämlich den gleichen Zustand wie heute.
Denn auch wenn die Bibliothek die Lizenzierung der Zeitschriften zentral
regelt, so unterschreiben heute WissenschaftlerInnen individuell für
jede ihrer Publikationen ein Copyright Transfer Agreement. Dies ist die
derzeit gelebte Realität. Und damit hört es in der Subskriptionswelt
noch nicht auf. Sollen beispielsweise einzelne Abbildung nochmals
veröffentlicht werden, so müssen die WissenschaftlerInnen diese Rechte
erwerben oder zumindest die erneute Veröffentlichung dem Verlag
individuell melden.\footnote{Laut der STM Association verlangen etwa
  drei viertel der Mitgliedsverlage zumindest eine Benachrichtigung über
  die erneute Veröffentlichung.
  \url{http://www.stm-assoc.org/copyright-legal-affairs/permissions/permissions-guidelines/}}
Das kostet manchmal Geld, vor allem aber verursacht es Bürokratie beim
Nachhalten der erworbenen Rechte und es kostet Zeit. Haben hingegen die
Transformationsverhandlungen Erfolg oder wandelt sich die
Publikationswelt auf einem anderen Weg hin zu Open Access, so entfällt
dieser Aufwand völlig. Aber auch die von Ball gefürchtete
Einzelabrechnung der AutorInnen mit Verlagen wird nicht Realität werden.
Schon heute sind ohne große Transformationen Rahmenverträge mit Verlagen
möglich. Die Arbeit für WissenschaftlerInnen wird also weniger, nicht
mehr. Bibliotheken haben das Bestreben, ihren NutzerInnen möglichst
wenig Arbeit zu bereiten. Genau das wird mit der Transformation zu Open
Access gelingen.

\hypertarget{transformationsvertruxe4ge-und-deal}{%
\subsubsection{Transformationsverträge und
DEAL}\label{transformationsvertruxe4ge-und-deal}}

Gegen Transformationsverträge werden einige Vorbehalte genannt:

\begin{quote}
\emph{\enquote{Durch den Abschluss von Read und Publish-Verträgen mit
den Großverlagen wird die Konzentration auf einige wenige Player noch
verstärkt. Die aktuelle Open Access Diskussion fokussiert dabei in
erster Line {[}sic!{]} auf den STM Sektor mit einigen wenigen großen
Playern in diesem Feld. {[}\ldots{}{]} Die Konzentration auf einige
wenige Monopolisten wird verstärkt und die Diversifizierung des Marktes
behindert.}}
\end{quote}

Dies ist unzutreffend. Zwar wird bei DEAL derzeit tatsächlich nur mit
den drei großen Verlagen Elsevier, SpringerNature und Wiley verhandelt;
parallel finden aber auch Gespräche mit kleinen und mittleren Verlagen
statt und sind zum Teil auch schon zum Abschluss gebracht, so zum
Beispiel im Rahmen der DFG-Ausschreibung
\enquote{Open-Access-Transformationsverträge}. Selbst um kleinste
Verlage kümmern sich Bibliotheken wie die der AutorInnen und
unterstützen bei der Transformation in den Open Access.\footnote{Vgl.
  \url{http://fz-juelich.de/zb/vzj}} Die in den letzten 20 Jahren
gegründeten Open-Access-Verlage wie Copernicus, MDPI, PLOS und Hindawi
haben zu neuen Playern geführt, die die Marktkonzentration zumindest
verlangsamen.

\begin{quote}
\emph{\enquote{Big Deals sind in der Summe noch größer und umsatzstärker
als es die alten Big Deals der Subskriptionen waren. Denn \enquote{Read
and Publish} ist logischerweise teuer als nur \enquote{Read}.}}
\end{quote}

Es handelt sich hierbei wohl um Erfahrungswissen, welches auf den in den
letzten Jahren abgeschlossenen Verträgen basiert. Bei diesen stieg der
Gesamtpreis in der Regel schon deshalb, weil neue Teilnehmer hinzukamen
und nicht alle Bestandsteilnehmer schon Zugriff auf alle Inhalte hatten.
DEAL zielt dagegen auf einen \enquote{Publish and Read}-Vertrag, bei dem
die Bezahllogik umgekehrt wird. Bezahlt werden die
Corresponding-Author-Publikationen aus den Teilnehmereinrichtungen; der
lesende Zugriff ist kostenfrei. Unter dieser Rücksicht ist
\enquote{Publish and Read} nicht notwendigerweise teurer als nur
\enquote{Read}. Und übrigens erschöpft sich das derzeitige
\enquote{Read}-Modell auch nicht in der bloßen Bezahlung der
Subskriptionsgebühren. Vielfach werden noch andere Gebühren, zum
Beispiel für Farbabbildungen oder Überlänge, erhoben und es müssen auch
noch Gebühren für die Nachnutzung an Verlage beziehungsweise das
Copyright Clearing Center\footnote{{[}Anm. d. Red.: Das Copyright
  Clearance Center (CCC) ist ein in den USA ansässiges Unternehmen,
  welches mit zahlreichen Verlagen kooperiert und verschiedene
  Dienstleistungen im Bereich Lizenzierung wissenschaftlicher Inhalte
  anbietet, so etwa Dokumentenlieferdienste oder Unterstützung für die
  automatisierte Einräumung von Nutzungsrechten und Abrechnung auf
  Einzelfallbasis. Für mehr Informationen s.
  \url{http://www.copyright.com/}.{]}} entrichtet werden. In einer
Open-Access-Welt mit CC-BY-Publikationen entfielen diese versteckten
Kosten.

Wie in einem Whitepaper der MPDL festgestellt und von Ball zitiert, sind
genügend finanzielle Mittel für die Literaturversorgung im
Wissenschaftssystem vorhanden.\footnote{Schimmer, Ralf; Geschuhn, Kai
  Karin \& Vogler, Andreas (2015): Disrupting the subscription journals'
  business model for the necessary large-scale transformation to open
  access. \url{https://doi.org/10.17617/1.3}} Leider wird diese Aussage,
die sich auf das Gesamtsystem bezieht, im Weiteren falsch interpretiert:

\begin{quote}
\emph{\enquote{{[}Diese Aussage{]} gilt ausnahmslos für einige wenige
gut ausgestattete Forschungsgemeinschaften und Eliteuniversitäten. Sie
ist leider nicht richtig für die große Masse der betroffenen Hochschulen
und Forschungseinrichtungen.}}
\end{quote}

Dies wird belegt\footnote{Osgyan, Verena (2015): Eklatante
  Unterfinanzierung der Erlanger und Nürnberger Hochschulbibliotheken
  muss ein Ende haben. Pressemitteilung, 18. November 2018. Online
  verfügbar unter
  \url{http://blog.osgyan.de/eklatante-unterfinanzierung-der-erlanger-und-nuernberger-hochschulbibliotheken-muss-ein-ende-haben/}}
mit einer der vielen Publikationen, die eine Unterfinanzierung von
Bibliotheken beklagen. Speziell diese hier zu zitieren ist nicht eben
naheliegend und jedenfalls Balls Behauptung nicht stützend, weil die
Unterfinanzierung im Kontext der steigenden Studierendenanzahl gesehen
wird. Wie viel aber publizieren StudentInnen? Eine Verbindung zu Open
Access wird dagegen nicht gezogen. Eigentlich meint die ursprüngliche
Aussage der MPDL, dass die bisherigen Lizenzzahlungen bei weitem dafür
ausreichen, alle notwendigen Leistungen der Verlage zu finanzieren,
weshalb der Umstieg auf Open Access keine Mehrkosten verursachen kann.

\begin{quote}
\emph{\enquote{Ganz ähnlich wie der permanente massive Preisanstieg der
wissenschaftlichen Zeitschriften Mitte der 1990er Jahre zur sogenannten
Zeitschriftenkrise und deren Folgen geführt hat (etwa der Entstehung von
Open Access Initiativen) besteht nun ein erhebliches Risiko, in eine
analoge Abhängigkeitssituation und Preisspirale durch permanente
Erhöhungen der APCs der Monopolisten zu geraten. Es ist nur schwer
verständlich, warum bei den meisten, aktuellen
Transformationsverhandlungen ebendiese sehr große (und
hochwahrscheinliche) Gefahr weder diskutiert noch berücksichtigt zu
werden scheint. Und dies verwundert umso mehr, als jene erfahrenen
Bibliothekare die Verhandlungen begleiten und mitgestalten, die die
große Zeitschriftenkrise als Konsequenz der Marktkonzentration zumeist
aktiv miterlebt haben.}}
\end{quote}

Was ist die Alternative zu Transformationsverhandlungen? Weiter
Subskriptionen mit altbekannter Preisspirale? Die am Ende des Beitrags
von Ball genannten Lösungsansätze adressieren dieses Problem allesamt
nicht. Dagegen sind bei Transformationsverträgen APCs erstmals
Gegenstand von Verhandlungen; die angestrebten Konditionen liegen
unterhalb der derzeitigen Hybrid OA-Preise. Naturgemäß kann man die
Konditionen nur für den Vertragszeitraum festlegen. Insofern bleibt für
die Zeit danach eine Unsicherheit. Aber mit Verträgen hat man zumindest
für deren Laufzeit Gewissheit -- ohne Verträge können Verlage völlig
nach Gusto agieren. Noch einmal, zur Verdeutlichung, das schon gebrachte
Zitat, im längeren Zusammenhang.

\begin{quote}
\emph{\enquote{Darüber hinaus bedeutet die Fixierung auf das Publizieren
bei einigen wenigen Verlagen durch eine vorhandene Flatrate der APCs
(wie sie etwa in den Niederlanden abgeschlossen worden ist) eine
durchaus diskutierbare Einschränkung der Freiheit von Forschung und
Wissenschaft. Wenn der Wissenschaftler oder die Wissenschaftlerin das
Publikationsorgan nicht mehr auswählen darf, das er oder sie für
fachlich geeignet hält, sondern nur aus der Liste derjenigen
Zeitschriftentitel auswählen darf, für die eine APC-Flatrate vereinbart
wurde, wird das gewiss als Einschränkung der Wissenschaftsfreiheit
interpretiert und empfunden werden und teilweise auch faktisch so sein.
{[}\ldots{}{]} Bei den Spitzenuniversitäten dieser Welt wird dieses
Thema eher eine Abstimmung mit den Füßen werden. Spitzenwissenschaftler
werden künftig bei ihren Berufungsentscheidungen Freiheiten oder
Einschränkungen ihrer jeweiligen Publikationsmöglichkeiten mit
berücksichtigen.}}
\end{quote}

Die \enquote{Liste} bezieht sich vermutlich auf den Einzelfall des
Offsetting-Vertrages 2016--2018 der Niederlande mit Elsevier.\footnote{\url{https://www.elsevier.com/about/open-science/open-access/agreements/VSNU-NL}}
Der Vertrag sieht vor, dass 30\,\% der niederländischen Publikationen
bei Elsevier im Jahr 2018 Open Access werden (2016 und 2017 noch
weniger). Dies wird geregelt durch eine Limitierung der hierfür in Frage
kommenden Zeitschriften. Dies bedeutet übrigens keineswegs, dass
AutorInnen bei der Auswahl des Publikationsorgans auf diese
Zeitschriften beschränkt wären. Sie können auch in anderen Zeitschriften
publizieren, nur dann eben nicht (ohne Mehrkosten) Open Access. Im
Übrigen ist das, wie erwähnt, ein Einzelfall: Die Verträge der
Niederlande mit anderen Verlagen\footnote{\url{https://www.vsnu.nl/en_GB/public-access-request}}
sehen das nicht vor; auch der nächste Elsevier-Vertrag der Niederlande
wird anders aussehen\footnote{Becking, Koen (2018): \enquote{Ook een
  \enquote{no deal} is mogelijk}. Online verfügbar unter
  \url{https://www.scienceguide.nl/2018/01/ook-no-deal-is-mogelijk/}}
und bei DEAL wird ein solcher Vertrag nicht abgeschlossen werden.
Selbstverständlich kann man nach wie vor auch bei anderen Verlagen
publizieren. Es ist skurril, einerseits jeglichen Abschluss von Publish-
and Read-Verträgen zu kritisieren und andererseits zu kritisieren, dass
es Publish- and Read-Verträge (noch) nicht mit allen Verlagen gibt. Und
was die Abstimmung mit den Füßen anbelangt -- was ist attraktiver: Eine
Universität, bei der man Zugang zu einem ausgewählten Portfolio an
Zeitschriften hat oder eine Universität, bei der man Zugang zu allen
Zeitschriften hat und außerdem noch ohne zusätzliche Kosten sämtliche
Artikel im Gold Open Access publizieren kann?

\hypertarget{aufgaben-der-bibliotheken}{%
\subsubsection{Aufgaben der
Bibliotheken}\label{aufgaben-der-bibliotheken}}

\begin{quote}
\emph{\enquote{Diese Kernaufgabe von Bibliotheken {[}quantitativ,
qualitativ und anspruchsgruppenspezifisch überprüfte Literaturauswahl{]}
wird überwiegend von den Teams der Fachreferenten (subject specialists)
geleistet. Sie wählen und beschaffen die Literatur nicht zufällig oder
aus Kostengründen, sondern auf der Basis der jeweiligen
Forschungsschwerpunkte und Lehrinhalte der Hochschulen und unter der
Berücksichtigung der jeweiligen Sammlungsstrategie der Bibliothek.}}
\end{quote}

Ball zeichnet hier ein sehr traditionelles Bild der FachreferentInnen.
In vielen Vorträgen und Publikationen hatte er in vergangenen Jahren
interessanterweise ein wesentlich moderneres Bild gezeichnet: So
beschrieb er im Jahr 1999 in einem Aufsatz\footnote{Ball, Rafael (1999):
  Die Diversifizierung von Bibliotheksdienstleistungen als
  Überlebensstrategie, BIT online 1/1999. Online verfügbar unter
  \url{http://www.b-i-t-online.de/archiv/1999-01/fachbeitraege/beitrag01/01.htm}}
zunächst unter der Zwischenüberschrift \enquote{Traditionelle
Kernaufgaben von Bibliotheken} diese mit den Worten: \enquote{\emph{Die
fachliche Auswahl der Literatur ist dabei bestimmt von der Art der
Bibliothek, ihrer Größe, ihrer rechtlichen Grundlage (und damit von
ihrem Auftrag), der zugeordneten Benutzergruppe und der möglichen
fachlichen Eingliederung in Sondersammelgebiete. Die fachliche Auswahl
der Literatur wird entweder durch Wissenschaftler der Fachbereiche oder
durch Fachreferenten geleistet}.}, um sie im Anschluss dann für überholt
zu erklären: \enquote{\emph{Die traditionelle Funktion der
Fachreferenten wird in zunehmenden Maße obsolet. Die Auswahl von
Literatur geschieht schon jetzt in vielen Bibliotheken durch
Buchhandlungen aufgrund exakter Profilangaben der Wissenschaftler. Auch
die Bestellung von Literatur durch Wissenschaftler selbst bedarf keiner
weiteren sachlichen Prüfung. Formalbibliothekarische Kontrollen, etwa
auf mögliche vorhandene Dubletten, andere Ausgaben, Nachdrucke und
andere Auflagen können von Diplombibliothekaren durchgeführt werden. Die
traditionelle Schwerpunktaufgabe eines wissenschaftlichen
Fachreferenten, die sachliche Erschließung von Literatur, wird immer
weniger wichtig, da moderne elektronische Suchsysteme das Auffinden und
Recherchieren von Literatur für die Benutzer auch ohne sachliche
Erschließung erleichtern. Es stehen zudem weitere elektronische Systeme
zur Verfügung, die durch automatisierte Indexierungssysteme und
formalisierte Logiken intellektuelle Erschließungsarbeiten der
Fachreferenten überflüssig machen. Auch die Entwicklung von Thesauri und
von übergeordneten Suchsystemen und -systematiken ist im Zuge des
Einsatzes moderner elektronischer integrierter Bibliothekssysteme
obsolet geworden}.} Vor knapp 20 Jahren erhielt Ball dafür einigen
Gegenwind; inzwischen ist dies weitgehend State-of-the Art.\footnote{Rothe,
  Ulrike; Johannsen, Jochen \& Schäffler, Hildegard (2014): Strategien
  des Bestandsaufbaus in der hybriden Bibliothek. In: Griebel, Rolf;
  Schäffler, Hildegard \& Söllner, Konstanze: Praxishandbuch
  Bibliotheksmanagement. Berlin/Boston: Walter de Gruyter} Oder mag sich
jemand vorstellen, dass eine Bibliothek wie die ETH-Bibliothek, die
bekanntlich von Ball geleitet wird, tatsächlich aus den 2.500
Elsevier-Zeitschriften einzelne Abonnements anhand des Forschungsprofils
ihrer mehr als 500 ProfessorInnen auswählt? Nein, sie lizenziert einfach
die Freedom Collection.

\begin{quote}
\emph{\enquote{Es kommt nicht von ungefähr, dass Bibliotheken bislang
ausgewählt haben und nur diejenige Literatur in ihrem Bestand halten und
nachweisen, die an der jeweiligen Hochschule benötigt wird oder zum
definierten Bestandsprofil passt. Der Zugriff aller auf alles ist nicht
automatisch der große Gewinn, sondern kann schnell zu einem Überangebot
werden, das mit großem Aufwand reduziert und fokussiert werden muss.}}
\end{quote}

Erneut: Das ist zuallererst ein Argument gegen Big Deals, nicht gegen
Transformationsverträge. Wir bezweifeln aber ohnehin, dass Bibliotheken
damit ein Problem haben. In der Elektronischen Zeitschriftenbibliothek
(EZB)\footnote{\url{https://ezb.uni-regensburg.de}} sind derzeit über
95.000 Zeitschriften nachgewiesen. 63\,\% davon sind jetzt schon frei
verfügbar; im Fall des Forschungszentrums Jülich sind weiterhin 13\,\%
durch Lizenzierung verfügbar und 24\,\% nicht verfügbar. Drei viertel
des Gesamtangebots ist also jetzt schon verfügbar und muss -- wenn man
das denn tun will -- \enquote{\emph{mit großem Aufwand reduziert und
fokussiert}} werden. Bei einem Komplettumstieg auf Open Access würde
sich dies um ein Drittel des bisherigen Wertes erhöhen. Das erscheint
verkraftbar.

\begin{quote}
\emph{\enquote{Das Thema Bestandsaufbau ist allerdings für die
Bibliotheken dann beendet, wenn durch große Verlagsflatrates die
\enquote{All-Inclusive-Versorgung} mit Literatur erreicht worden ist.
Damit endet dann auch eine der klassischen Kernaufgaben von
Bibliotheken, nämlich die Aufbereitung und Strukturierung von
Informationen.}}
\end{quote}

Nach klassischer Definition ist die Bibliothek \enquote{\emph{eine
Einrichtung, die unter archivarischen, ökonomischen und synoptischen
Gesichtspunkten publizierte Information für die Benutzer sammelt, ordnet
und verfügbar macht}.}\footnote{Ewert, Gisela: Lehrbuch der
  Bibliotheksverwaltung / auf der Grundlage des Werkes von Wilhelm
  Krabbe und Wilhelm Martin Luther völlig neu bearb. von Gisela Ewert
  und Walther Umstätter. Stuttgart: Hiersemann, 1997, S. 10} Warum
sollte das wegfallen, wenn die Literatur Open Access verfügbar ist?

\begin{quote}
\emph{\enquote{Die Methoden der Wahl sind dabei die Formalerschließung
(Katalogisierung), sowie die Sacherschließung und Beratung. Diese
zentralen Leistungen haben dem Benutzer einen schnellen, fokussierten,
sachgerechten und unabhängigen Zugang zur gewünschten Literatur des
Fachgebiets oder des gewünschten Themas geboten. Wenn nach der
Umstellung und der Transformation des Publikationssystems sämtliche
Literatur für alle kostenlos verfügbar sein wird, wird es Bibliotheken
kaum mehr gelingen eine fachlich fokussierte Auswahl der Literatur zu
treffen, sie mit entsprechend großen maschinellen oder intellektuellen
Aufwand zu erschließen, und den Zugang zu organisieren und für lange
Zeit zu garantieren.}}
\end{quote}

Die Katalogisierung erfolgt auch jetzt schon kollaborativ und
maschinengestützt (zum Beispiel Abzug aus der EZB). Sollte eine
intellektuelle Inhaltserschließung auf Artikelebene gemeint sein, so
wird diese nur von wenigen Spezialbibliotheken auf sehr begrenzten
Gebieten vorgenommen. Die betreffenden Einrichtungen wissen jetzt
auszuwählen und werden dies problemlos auch in Zukunft tun. Und auch
dies: Die Open-Access-Transformation selbst produziert keine neuen
Zeitschriften, sondern wandelt nur die bestehenden
Subskriptionszeitschriften um. Sie trägt damit sogar zu einer
Verlangsamung des Wachstums der Zeitschriftenzahl bei, weil derzeit
gerne Open-Access-Zeitschriften parallel zu bestehenden
Subskriptions-Titeln mit sehr ähnlichem Fokus gegründet
werden.\footnote{Als Beispiel seien hier die Zeitschriften
  \enquote{Biochimie} und \enquote{Biochimie Open} von Elsevier genannt.}

\begin{quote}
\emph{\enquote{Wenn Veröffentlichungen künftig direkt vom Autor durch
die APCs finanziert werden müssen, erfordert dies zudem eine ganz
besondere Art der budgetären Planung. Die Bereitstellung und
Bereithaltung von finanziellen Mitteln für potentielle
Veröffentlichungen ist im Unterschied zur (planbaren) Beschaffung,
Lizenzierung und Bezahlung von Literatur durch die Bibliothek aber ein
nahezu aussichtsloses Unterfangen.}}
\end{quote}

Dies ist keine \enquote{ganz besondere Art der budgetären Planung},
sondern der Normalfall. Auch bei Personalausgaben, Energiekosten,
Reisekosten et cetera weiß man am Jahresanfang nicht sicher, wieviel man
benötigt. Nur bei den Literaturetats war bislang der größte Teil schon
am Jahresanfang verausgabt.

\begin{quote}
\emph{\enquote{Denn Verfügbarkeit von Wissenschaftlerstellen,
Kreativität, der zeitliche Verlauf von (Labor) Experimenten und deren
Ergebnissen, sowie die nicht kalkulierbare Dauer des Peer
Review-Verfahrens bei den Verlagen und die damit zusammenhängende
Annahme oder Ablehnung eines Manuskripts machen eine mittelfristige
Budgetplanung für zu erwartende Publikationskosten, also die
Bereitstellung der APCs, praktisch unmöglich. Es muss deshalb befürchtet
werden, dass gerade die weniger finanzstarken Einrichtungen und
Hochschulen eines Landes bei der Finanzierung ihrer
Veröffentlichungsgebühren in ernste Schwierigkeiten geraten können. Es
wäre ein trauriges Resultat der Transformation des Publikationssystems,
wenn in vielen Hochschulen und Universitäten schon Mitte des Jahres die
Mittel zur Zahlung der APCs erschöpft wären.}}
\end{quote}

Wenn eine Bibliothek ihren bisherigen Zeitschriftenetat für
Open-Access-Publikationsgebühren bereit hält, dann wird dieser in den
allermeisten Fällen ausreichend sein\footnote{Schimmer, Ralf (2012).
  Open Access und die Re-Kontextualisierung des
  Bibliothekserwerbungsetats. Bibliothek, Forschung und Praxis 36(3),
  293--299 \url{https://doi.org/10.1515/bfp-2012-0038}} -- insbesondere
dann, wenn nicht verausgabte Mittel übertragen werden können, um auf
Fluktuationen im Publikationsaufkommen reagieren zu können. Ausgehend
vom bisherigen Publikationsaufkommen, wie es beispielhaft im Abschnitt
\enquote{Perspektive der AutorInnen} erhoben wurde, kann mit Hilfe von
Daten über die Höhe von Publikationsgebühren eine Korridorplanung der
Ausgaben für Publikationen gemacht werden. Dies beschreiben bereits
Geschuhn und Pieper in ihrem Projekt INTACT: \emph{\enquote{Mit der
integrierten Open-APC-Initiative können Kennzahlen über das
Publikationsaufkommen einzelner Einrichtungen mit den entsprechenden
Kosteninformationen verbunden werden. Damit entsteht eine
verlagsunabhängige, transparente empirische Basis {[}\ldots{}{]} auch
für nach erfolgter Open-Access-Transformation rein publikationsbezogene
Modelle.}}\footnote{Geschuhn, Kai Karin \& Pieper, Dirk (2016). Wandel
  aktiv gestalten: Das Projekt INTACT-Transparente Infrastruktur für
  Open-Access-Publikationsgebühren. In \emph{Schriften des
  Forschungszentrums Jülich, Reihe Bibliothek / Library}, 22, Seiten
  47--69. WissKom 2016, Jülich. \url{http://hdl.handle.net/2128/11559}.
  Hier Seite 66.}

Zu berücksichtigen ist auch, dass für Publikationsausgaben viel eher als
für Literaturbeschaffung im Rahmen von Drittmittelprojekten zusätzliche
Mittel eingeworben werden können, die zur Verfügung stehenden Mittel
somit über den Bibliotheksetat hinaus anwachsen. Zudem ist darauf
hinzuweisen, dass die Behauptung von Ball \enquote{\emph{Die Bezahlung
der APCs ist alternativlos}} verkennt, dass viele OA-Zeitschriften ein
anderes Finanzierungsmodell haben. Tatsächlich erheben 70\,\% der im
Directory of Open Access Journals verzeichneten Open
Access-Zeitschriften keine APCs.\footnote{Laut dem Directory of Open
  Access Journals (Stand 7. März 2018) erheben 3.186 der Zeitschriften
  APCs und 7.959 keine. Bei 102 der Zeitschriften gibt es keine
  Informationen dazu. \url{https://doaj.org}} Sollten die Mittel also
tatsächlich zu Neige gehen, kann immer noch in einer dieser
Open-Access-Zeitschriften publiziert werden. Aber beide
Ausweichmöglichkeiten wären in der Praxis vermutlich auch nicht zwingend
notwendig, denn die Sicherstellung des wissenschaftlichen Outputs wird
eine Einrichtung stets mit sehr hoher Priorität versehen.

\hypertarget{verhuxe4ltnis-zu-den-verlagen}{%
\subsubsection{Verhältnis zu den
Verlagen}\label{verhuxe4ltnis-zu-den-verlagen}}

\begin{quote}
\emph{\enquote{Gerade im STM Segment wird die Abhängigkeit der
Literaturversorgung und des Publizierens von einigen Großverlagen und
die damit einhergehende Konzentration des Marktes zurecht beklagt.}}
\end{quote}

Mit dieser Feststellung hat Ball nicht Unrecht. Die Konzentration auf
dem Publikationsmarkt auf wenige Großverlage\footnote{Larivière,
  Vincent; Haustein, Stefanie \& Mongeon Philippe (2015), The Oligopoly
  of Academic Publishers in the Digital Era. PLoS ONE 10(6): e0127502.
  \url{https://doi.org/10.1371/journal.pone.0127502}} hat sich
allerdings in einer Zeit entwickelt, in der das Publizieren in
Subskriptionszeitschriften die Regel war und ist nicht in der Umstellung
auf Open Access begründet. Im Gegenteil ist diese in der Zeit der
Subskriptionszeitschriften entstandene Konzentration auch einer der
Gründe dafür, dass die Verhandlungen von Transformationsverträgen vor
allem bei den Großverlagen ansetzten. Mit einem Abschluss der
DEAL-Verträge wird ermöglicht, dass Mittel für Verträge mit kleinen und
mittleren Verlagen frei werden. Ball bleibt in seiner Kritik an den
\enquote{Big Deals} jedoch bei den großen STM-Verlagen verhaftet und
beklagt zwar, die Perspektive auf die kleinen und mittleren Verlage
würde versäumt, nimmt diese aber selbst nicht in den Blick. Ohne große
Mühe finden sich bei einfacher Recherche auch kleinere Initiativen und
Verlage, die das Thema Open Access erfolgreich umsetzen. Ein Beispiel
ist die Electrochemical Society (ECS), für die es seit 2018 im Rahmen
eines Nationalkonsortiums einen Transformationsvertrag gibt. Es handelt
sich dabei um ein kostengünstiges Read-and-Publish-Modell.\footnote{\url{https://www.goportis.de/fileadmin/downloads/lizenzen/TIB/Konsortien/ECS_Plus_Datenblatt_2018.pdf}}
Ziel der ECS ist eine komplette Umstellung auf Open Access.\footnote{\url{http://ecsdl.org/site/misc/oa.xhtml}}

Zur verlagsunabhängigen Bereitstellung der Literatur in den Bibliotheken
hat Ball merkwürdige Vorstellungen, wie er sie im auch schon einmal in
diesem Text gebrachten Zitat darstellt:

\begin{quote}
\emph{\enquote{Damit endet dann auch eine der klassischen Kernaufgaben
von Bibliotheken, nämlich die Aufbereitung und Strukturierung von
Informationen. Dabei ist es im Zusammendenken mit der Erwerbungsauswahl
eine zentrale Aufgabe von wissenschaftlichen Bibliotheken, einen
Überblick über die Vielzahl der verschiedensten Informationen und
Literaturangebote für die jeweiligen Anspruchsgruppen zu organisieren
und zu strukturieren -- und zwar unabhängig und neutral von der
jeweiligen Verlagsplattform, dem Verlag oder dem Erscheinungsmedium.
{[}\ldots{}{]} Wenn nach der Umstellung und der Transformation des
Publikationssystems sämtliche Literatur für alle kostenlos verfügbar
sein wird, wird es Bibliotheken kaum mehr gelingen eine fachlich
fokussierte Auswahl der Literatur zu treffen, sie mit entsprechend
großen maschinellen oder intellektuellen Aufwand zu erschließen, und den
Zugang zu organisieren und für lange Zeit zu garantieren.}}
\end{quote}

Es fällt schwer, die Argumentation in diesem Abschnitt nachzuvollziehen,
wonach der Anspruch, Zugang zu Informationsangeboten unabhängig von
Plattformen und Erscheinungsmedien zu organisieren, angeblich einfacher
sein soll, wenn wirtschaftliche Zwänge diesen Zugang einschränken. Im
Gegenteil bietet doch die Zunahme von frei verfügbarer Literatur
deutlich mehr Möglichkeiten, eine rein fachlich fokussierte Auswahl für
die Bereitstellung und Aufbereitung in Bibliotheksplattformen zu
treffen. Ein Kriterium, das Ball schließlich auch für die Auswahl der
Zeitschrift beim Publizieren einfordert, was die Argumentation hier noch
skurriler wirken lässt. Sicherlich stellt die Erschließung und
Bereitstellung frei verfügbarer Quellen eine Herausforderung dar -- eben
eine neue Aufgabe für FachreferentInnen im Rahmen ihres Kerngeschäfts.
Ein Beispiel dafür, wie Bibliotheken an diese Aufgabe herangehen, findet
sich übrigens im selben Heft von BIT online, in dem auch Balls Artikel
erschienen ist. In der Reportage \enquote{Zukunft Bibliotheken} wird
dargestellt, wie die SUB Bremen in Kooperation mit der UB Bielefeld
OA-Publikationen in ihrem Rechercheportal nachweist.\footnote{Münch,
  Vera (2018): Zukunft Bibliothek -- weiter auf neuen Wegen. B.I.T.
  online 21(1), 58--63. Online verfügbar unter
  \url{http://www.b-i-t-online.de/heft/2018-01-reportage-muench.pdf}}

\begin{quote}
\emph{\enquote{Ein besonderes Augenmerk (auch das scheint aus dem Blick
der großen Big Deal-Verhandlungen zu geraten) ist die politische und
wirtschaftliche Unabhängigkeit des Bestandsaufbaus, der Erschließung und
Vermittlung von Informationen durch Bibliotheken. Open Access verfügbare
Literatur wird künftig alleine über die jeweiligen (mächtigen),
partikulären Plattformen der Verlagsindustrie zur Verfügung gestellt
werden. Eine verlagsunabhängige Aufbereitung und Suche von
wissenschaftlicher Literatur wird vor diesem Hintergrund für
Bibliotheken immer schwerer, wenn nicht sogar unmöglich.}}
\end{quote}

Ball geht hier auf die gewünschte Unabhängigkeit der Bibliotheken
gegenüber den Verlagen ein, die er anscheinend durch ein
kostenpflichtiges Angebot eher gesichert sieht als bei kostenfreier
Nutzung. Auch in diesem Abschnitt wird das Muster verfolgt, waghalsige
Behauptungen mit abenteuerlichen Schlussfolgerungen zu kombinieren.
Gerade im Subskriptionsmodell ist es doch die Regel, dass die Literatur
ausschließlich auf den verlagseigenen Servern bereitgestellt wird. Die
Anbieter von Subskriptionszeitschriften erlauben zwar meist das
Erstellen von Sicherungskopien der Inhalte, eine Bereitstellung für die
Nutzer ist vertraglich aber nur durch Verlinkung auf den jeweiligen
Verlagsserver gestattet.\footnote{Öffentlich einsehbar ist zum Beispiel
  der Vertrag der Niederlande mit Wiley: \enquote{The Licensee and its
  Authorized Users may create links to Wiley Online Library from their
  Online Public Access Catalog (OPAC) records, libray catalogs, link
  resolvers, locally hosted databases or library web pages
  {[}\ldots{}{]}}
  \url{https://www.vsnu.nl/files/documenten/Domeinen/Onderzoek/Open\%20access/Wiley.pdf}}
Im Gegensatz dazu ist durch die im Open Access vorliegenden
CC-BY-Lizenzen den Bibliotheken die Möglichkeit gegeben, die Literatur
auf eigenen Servern zu speichern. Und wie das Beispiel der SUB Bremen
zeigt, ist eine verlagsunabhängige Bereitstellung von Literatur
selbstverständlich gerade bei Open Access möglich, indem über die
Rechercheportale der Bibliotheken unter anderem auf die in Repositorien
liegenden Volltexte (Verlags-PDF) verlinkt wird.

\hypertarget{langzeitarchivierung}{%
\subsubsection{Langzeitarchivierung}\label{langzeitarchivierung}}

\begin{quote}
\emph{\enquote{Die Verträge der Bibliotheken mit den Verlagen über die
Lieferung von Inhalten (Subskription/Lizenzierung) beinhalten (meist)
auch eine Zusicherung der Archivierung der Inhalte in unabhängigen
Strukturen und Organisationen (etwa einer Nationalbibliothek). Damit
wird sichergestellt, dass die Inhalte auch dann zur Verfügung stehen,
wenn der Verlag nicht mehr existiert, sein Programm sich ändert oder
aber andere Produkte hergestellt werden. In Zukunft werden Verlage bei
der Konzeption der Geschäftsmodelle (Author-Pays-Modell) und der
Finanzierung des Systems über die APCs keinen großen Wert mehr auf eine
garantierte Archivierung der Inhalte legen, da sie das Geld bereits mit
der Veröffentlichung verdient haben. {[}\ldots{}{]} Archivierung und
Langzeitverfügbarkeit von Inhalten stehen vor diesem Hintergrund vor
einer großen Herausforderung. Man kann sogar so weit gehen und vermuten,
dass sie durch die Transformation des Publikationssystems nicht mehr
gesichert sind.}}
\end{quote}

Wieder stellt Ball eine These ohne Verifizierung auf -- es gibt
schließlich auch keine Belege dafür, dass eine Sicherung von Inhalten in
unabhängigen Strukturen in einer Open-Access-Welt nicht mehr gegeben
sein würde. Er müsste im Gegenteil wissen, dass zunächst durch die
Pflichtabgabeverordnung\footnote{\url{http://www.gesetze-im-internet.de/pflav/index.html}}
bei der Deutschen Nationalbibliothek grundsätzlich eine Sicherung
gegeben ist. Die Sammelrichtlinien\footnote{\url{https://d-nb.info/1051940788/34}}
unterscheiden nämlich nicht zwischen OA und nicht-OA. Und auch in der
Schweiz, in der Ball tätig ist, erfolgt eine solche Pflichtabgabe --
nach ISO-Norm.\footnote{\url{https://www.nb.admin.ch/snl/de/home/nb-professionell/e-helvetica/faq/faq-zu-den-arbeitsablaeufen.html}}
Die British Library hat einen generellen gesetzlichen Auftrag zur
Sammlung elektronischer Publikationen und erledigt dies bei frei
verfügbaren Ressourcen durch automatisiertes Harvesting.\footnote{\url{http://www.bl.uk/aboutus/legaldeposit/websites/elecpubs/}}
Konzepte zur Langzeitarchivierung existieren zum Beispiel für
OJS-gehostete Zeitschriften\footnote{\url{https://pkp.sfu.ca/2016/06/01/launch-of-private-lockss-network-for-ojs-journals/}}
und das Directory of Open Access Journals (DOAJ)\footnote{\url{https://doaj.org/}}
zeigt mit dem DOAJ Seal\footnote{\url{https://doaj.org/faq\#seal}} unter
anderem als Qualitätsmerkmal an, welche Zeitschriften ein entsprechendes
Konzept vorlegen. Das Thema ist also bereits in der Praxis bei den
Akteuren verankert und stellt im Übrigen eine grundsätzliche
Herausforderung dar. Das gilt auch jetzt schon für
Subskriptionszeitschriften -- deshalb gibt es LOCKSS\footnote{\url{https://www.lockss.org/}}
und Portico\footnote{\url{https://www.portico.org/}} oder auch das
Projekt \enquote{Nationales Hosting Elektronischer Ressourcen}
(NatHosting), das zum Ziel hat, eine \enquote{bundesweit abgestimmte
Strategie zur Lösung der {[}\ldots{}{]} Problematik für kommerziell
vertriebene Inhalte zu entwickeln}\footnote{\url{https://www.nathosting.de/display/ND/Hintergrund}}.
Ob Verlage das Geld durch APCs oder durch Subskriptionsgebühren
verdienen, dürfte auf die Praxis der Archivierung eher keinen Einfluss
haben. Hier überhaupt einen Zusammenhang zu erschließen, erscheint
fragwürdig.

\hypertarget{open-access-in-den-geisteswissenschaften-und-open-access-monographien}{%
\subsubsection{Open Access in den Geisteswissenschaften und
Open-Access-Monographien}\label{open-access-in-den-geisteswissenschaften-und-open-access-monographien}}

\begin{quote}
\emph{\enquote{Die massive Konzentration auf Big Deals im STM-Segment
verstärkt zugleich die Dominanz der digitalen Informationsversorgung.
Niemand will das Rad zurückdrehen und in den STM-Fächern ist die
elektronische Verfügbarkeit von Literatur und Information seit langem
erwarteter Standard. Doch in den Geistes- und Sozialwissenschaften
gehören gedruckte Medien noch immer zur relevanten
Informationsversorgung {[}\ldots{}{]}. Allein an deutschen Hochschulen
waren im Studienjahr 2016/17 mehr als die Hälfte aller Studierenden in
einem geistes-, sozial- oder wirtschaftswissenschaftlichen Studiengang
eingeschrieben {[}\ldots{}{]}. Die aktuelle OA Diskussion hingegen
ignoriert diesen nicht unbeträchtlichen Teil der Literaturversorgung an
Universitäten und Hochschulen völlig und macht die Geistes- und
Sozialwissenschaften zu Wissenschaften zweiter Klasse, deren
Literaturversorgung sich nur noch aus den Restmitteln speist, die nach
Abschluss der Big Deals übrigbleiben. Auch die (sinnvollen) Initiativen
zu Open Access von (digitalen) Monografien ignorieren dabei aber den
berechtigten und zu akzeptierenden Wunsch der Geistes- und
Sozialwissenschaften nach Nutzung und Veröffentlichung in gedruckten
Medien.}}
\end{quote}

Ball scheint die Diskussion zu Open Access in den Geisteswissenschaften
selbst nicht zu verfolgen. Daraus schließt er für sich, dass sie nicht
stattfindet. Dass die Bedingungen in den Geisteswissenschaften andere
sind als im STM-Bereich, ist sicherlich nicht zu bestreiten. Aber
entgegen Balls Behauptung, die Geisteswissenschaften würden in der
aktuellen OA-Diskussion völlig ignoriert, gibt es auch in diesem Bereich
etliche Initiativen, die durchaus bekannt und ebenso Belege für eine
stattfindende Diskussion sind, die im Folgenden genannt werden. Aus
dieser kristallisieren sich Unterschiede zum STM-Bereich beziehungsweise
zwischen Buch- und Zeitschriftenpublikationen heraus.

Die Publikation von Monographien ist essentiell für die
Geisteswissenschaften, da nicht eine reine Ergebnispräsentation von
Forschung im Vordergrund steht. Vielmehr benötigt die Entwicklung einer
geisteswissenschaftlichen Argumentation und der Bezug zu Originalwerken
einen Umfang, der in Zeitschriftenartikeln nicht abbildbar ist.
Forschende in den Geisteswissenschaften identifizieren sich zudem in
hohem Maße über ihre eigenen Buchveröffentlichungen, die für ihre
Karriere ebenso wesentlich sind wie Zeitschriftenveröffentlichungen es
für NaturwissenschaftlerInnen sind. \enquote{\emph{In vielen Fällen
versteht sich Geisteswissenschaft als Arbeit am Werk. In nicht wenigen
verstehen sich die Forschenden auch {[}\ldots{}{]} selbst als
Werkschöpfer.}}\footnote{Kaden, Ben (2017): Publikationsfreiheit.de,
  Open Access und Geisteswissenschaften. LIBREAS.Library Ideas, 31. Juli
  2017. Online verfügbar unter
  \url{https://libreas.wordpress.com/2017/07/31/open-access_publikationsfreiheit/}}

Um Monographien Open Access publizieren zu können, werden andere
Geschäftsmodelle diskutiert,\footnote{\url{http://oad.simmons.edu/oadwiki/OA_book_business_models}}
und müssen andere Probleme als im STM-Bereich adressiert werden, wie zum
Beispiel die höheren entstehenden Kosten, die durch die geringeren
Budgets im geisteswissenschaftlichen Bereich schwer zu decken sind,
Lizenzfragen bei der Mit-Veröffentlichung von Werken, auf die sich die
jeweilige Forschung bezieht, und die technische Umsetzung. Andererseits
werden auch die Chancen von Open-Access-Modellen und digitaler
Veröffentlichung für Monographien in der Diskussion benannt. Da ist zum
einen die Möglichkeit, geisteswissenschaftliche Texte durch die digitale
Veröffentlichung anzureichern, zum Beispiel mit Suchfunktionen,
Einbettung zusätzlicher Medien, Verlinkung zu anderen Texten und
Bereitstellung von Kommentarfunktionen. Des Weiteren garantiert die
freie Verfügbarkeit für NutzerInnen den Zugang zum kompletten Text, was
bei der digitalen Bereitstellung von Monographien die Integrität des
Werkes sichert. Werden Monographien dagegen kostenpflichtig digital
bereitgestellt, werden wirtschaftliche Zwänge den Download nur einzelner
Kapitel begünstigen.\footnote{Crossick, Geoffrey (2016): Monographs and
  open access. Insights 29(1), 14--19. Online verfügbar unter
  \url{https://insights.uksg.org/articles/10.1629/uksg.280/}} Hinzu
kommt das Argument, dass durch Open-Access-Veröffentlichungen eines
Buches sogar die Absatzzahlen für die gedruckte Version erhöht werden
können, indem Aufmerksamkeit für das Werk geweckt wird.\footnote{Kohle,
  Hubertus (2013): Für Open Access in den Geisteswissenschaften.
  perlentaucher.de. Das Kulturmagazin, 19. März 2013. Online verfügbar
  unter
  \url{https://www.perlentaucher.de/essay/fuer-open-access-in-den-geisteswissenschaften.html}}
Open Access kann so als Marketinginstrument dienen.\footnote{\url{https://open-access.net/informationen-fuer-verschiedene-zielgruppen/verlage/}}
Der Bedarf nach gedruckten Büchern bleibt also nicht unbeachtet. Das
Directory of Open Access Books (DOAB)\footnote{\url{https://www.doabooks.org/doab?func=publisher}}
listet über 260 Verlage mit dem Nachweis ihrer Open-Access-Monographien
-- mit Links sowohl zur frei zugänglichen digitalen Version als auch zu
Bezugsmöglichkeit für Print-Exemplare. Den Bedarf nach gedruckten Medien
hat Ball selbst in seinem Interview in der NZZ (\enquote{Weg mit den
Büchern})\footnote{Ball, Rafael (2016): Bibliotheken: Weg damit! NZZ am
  Sonntag, 7. Februar 2016. Online verfügbar unter
  \url{http://www.nzz.ch/nzzas/nzz-am-sonntag/bibliotheken-weg-mit-den-buechern-interview-rafael-ball-eth-ld.5093}}
vor zwei Jahren übrigens noch negiert.

Es gibt noch viele weitere Portale für Open-Access-Monographien,
beispielsweise die Open Library of Humanities\footnote{\url{https://www.openlibhums.org/}},
OAPEN\footnote{\url{http://www.oapen.org/home}} oder Knowledge
Unlatched\footnote{\url{http://www.knowledgeunlatched.org/}}, als kleine
Verlage zu nennen wären transcript\footnote{\url{http://www.transcript-verlag.de/transcript-in-open-access-netzwerken}},
oder\textsuperscript{,} Language Science Press\footnote{\url{http://langsci-press.org/}},
als mittelgroßer Verlag DeGruyter\footnote{\url{https://www.degruyter.com/dg/page/open-access-books}}.
Sie alle repräsentieren auch das Gebiet der Geisteswissenschaften. Das
von SpringerNature im November 2017 veröffentlichte White Paper
\enquote{The OA effect: How does Open Access affect the usage of
scholarly books?}\footnote{Emery, Christina; Lucraft, Mithu, Morka,
  Agata \& Pyne, Ros (2017): The OA effect: How does Open Access affect
  the usage of scholarly books? White paper. Online verfügbar unter
  \url{https://resource-cms.springer.com/springer-cms/rest/v1/content/15176744/data/v3}}
zur Nutzung von Open-Access-Büchern weist zudem auch in den
Geisteswissenschaften eine erhöhte Reichweite bei Open Access
veröffentlichten Monographien nach. Und schließlich informiert die
Plattform open-access.net in einer Fächerübersicht ausführlich zum
Themenspektrum auch in den Geisteswissenschaften.\footnote{\url{https://open-access.net/informationen-fuer-verschiedene-faecher/}}

In Anbetracht all dieser Angebote erscheint es geradezu unseriös, dass
Ball zwar die Bedürfnisse der Geisteswissenschaften in den Raum stellt,
diese aber weder benennt noch Quellen heranzieht, in denen diese
adressiert werden. So lässt er nur eine unglaubwürdige Behauptung
stehen, anstatt konkret zur Diskussion beizutragen.

\hypertarget{trittbrettfahrer}{%
\subsubsection{Trittbrettfahrer}\label{trittbrettfahrer}}

Auch wenn Ball meint, dass \emph{\enquote{selbst Wissenschaftler
verwandter Disziplinen {[}\ldots{}{]} mit der Fachliteratur der
Nachbardisziplinen schon nichts mehr anzufangen}} können, so sieht er
doch die Gefahr von Trittbrettfahrern.

\begin{quote}
\emph{\enquote{Ein näherer Blick und eine differenzierte Betrachtung
zeigen hingegen, dass vor allem kommerzielle Unternehmen, die nicht im
nennenswerten Umfang als Autoren in das System des Publizierens
investieren (also keine APCs bezahlen), ausschließlich zu den
Profiteuren der Transformation zählen dürfen. Dazu gehören etwa private,
forschungsstarke Unternehmen, wie die Pharmaindustrie, die biochemische
und chemische Industrie, aber auch Unternehmen des Maschinenbaus, der
Autoindustrie, Softwarefirmen, Banken und Versicherungen. Auch viele
kleinere und mittlere Betriebe wie Anwaltskanzleien, niedergelassene
Ärzte, Krankenhäuser oder Ingenieurbüros, die bisher für die Nutzung von
wissenschaftlicher Information bezahlt, und dies in ihrem Businessmodell
berücksichtigt haben, werden nun von der kostenlosen Verfügbarkeit
dieser Information profitieren, ohne dass sie das System selbst (durch
die Zahlung von APCs) unterstützen.}}
\end{quote}

Diese Befürchtung enthält einen wahren Kern, ist aber zu allgemein
formuliert. Richtig ist, dass die (bio-)chemische Industrie und die
Pharmaindustrie profitieren werden -- jedenfalls dann, wenn
Zeitschriften komplett umgestellt werden. Für Verhandlungen wie zum
Beispiel bei DEAL spielt dieser Effekt keine Rolle. Mittelfristig können
sich Fachgesellschaften wie die Gesellschaft Deutscher Chemiker (GDCh)
Gedanken machen, wie sie ihre Arbeit stärker durch direkte Beiträge der
chemischen Industrie finanzieren als indirekt über die Abonnements der
Angewandten Chemie und somit aus Steuergeldern. Anwaltskanzleien spielen
wohl auf absehbare Zeit keine Rolle, da juristische Literatur angesichts
der dort häufig gezahlten AutorInnenhonorare am wenigsten für die
Umstellung auf Open Access geeignet ist. Was niedergelassene ÄrztInnen
und Ingenieurbüros anbelangt, so ist die Zahl der abonnierten
wissenschaftlichen Zeitschriften angesichts der Preise ziemlich gering.
Als PatientInnen würden wir es uns dagegen sehr wohl wünschen, dass die
uns behandelnden ÄrztInnen Zugang zu jeglicher relevanter Information
hätten.

\hypertarget{zusammenfassung}{%
\subsubsection{Zusammenfassung}\label{zusammenfassung}}

Ball erkennt in seinem Beitrag eingangs den Wunsch nach Open Access als
durchaus sinnvoll an.

\begin{quote}
\emph{\enquote{Der Wunsch der Wissenschaftswelt nach freiem Zugang zu
wissenschaftlichen Informationen und Publikationen ist mehr als
verständlich, da für die Beschaffung von Informationen und Literatur
große finanzielle Mittel der öffentlichen Hand verwendet werden. Der
Wunsch, in einer wissenschaftlichen Zeitschrift nichts nur problemlos
publizieren (\enquote{easy to publish}), sondern deren Inhalte auch
weltweit kostenlos lesen zu können (\enquote{easy to read}) ist deshalb
nachvollziehbar.}}
\end{quote}

Es folgt dann allerdings eine Vielzahl irriger Annahmen und
Fehlinterpretationen, was zu falschen Darstellungen führt und umso
gravierender zu sein scheint, je weiter man in der Lektüre des Artikels
fortschreitet. Dies wird dann leider auch noch mit einer zweifelhaften
Rhetorik versehen:

\begin{quote}
\emph{\enquote{Ein populistisches -- ja bisweilen sogar ideologisches --
Herangehen wird die Reformation der Wissenschaftskommunikation, die
Öffnung und Verfügbarkeit ihrer Ergebnisse und den Abbau der
Abhängigkeit von monopolistischen Märkten eher behindern als fördern.
Sie ist auch nicht als kurzfristige \enquote{Revolution} innerhalb einer
Amtszeit von Rektoren und Hochschulpräsidenten umzusetzen, die das Kind
mit dem Bade ausschüttet, eine (noch) funktionierende Verlagsvielfalt
beendet und ein ungeordnetes Publikationschaos verursacht, das dann mit
viel Geduld und Ressourcen aufwendig repariert werden muss.}}
\end{quote}

Man fragt sich, worauf der Direktor der ETH Zürich-Bibliothek damit
hinaus will. Die Lösungsansätze, die er am Ende des Beitrags bietet,
sind jedenfalls keine Offenbarung. Zum größten Teil sind es
Selbstverständlichkeiten, die jetzt und in Zukunft gelten und die
niemand in Frage stellt. Allenfalls diskussionswürdig ist die Forderung
einer \enquote{\emph{Literatur- und Informationsversorgung in der
Verantwortung der Bibliotheken statt zentraler, nationaler Finanzierung
von Flatrates mit zweifelhaftem Nutzen}}. Auch abgesehen vom Aspekt des
angeblich \enquote{\emph{zweifelhaften Nutzen}s} der Flatrates, welcher
von den AutorInnen ausführlich diskutiert wurde, steht jedenfalls für
Deutschland eine zentrale, nationale Finanzierung gar nicht in der
Diskussion. Niemand wird zum DEAL oder zur Teilnahme an einem
DFG-geförderten Transformationsvertrag gezwungen, es bleibt stets
individuelle Entscheidung jeder Bibliothek. Ziel der Verhandlungsführung
ist ein Ergebnis, das die Teilnahme für möglichst viele attraktiv macht.
Diese Transformationsverträge sind keineswegs der einzige Weg zur
Umstellung des Publikationswesens in den Open Access. Dazu gehört das
erwähnte Flipping einzelner Zeitschriften ebenso wie die Gründung von
Gold-Open-Access-Zeitschriften\footnote{Holzke, Christoph; Frick,
  Claudia \& Mittermaier, Bernhard (2016): DOI-Vergabe für Großgeräte im
  Journal of Large Scale Research Facilities JLSRF.
  \url{http://hdl.handle.net/2128/17600}} und weitere, auch alternative
Denkweisen und Geschäftsmodelle.\footnote{Söllner, Konstanze \&
  Mittermaier, Bernhard (Hrsg.) (2017): Praxishandbuch Open Access.
  Berlin/Boston: Walter De Gruyter.
  \url{https://doi.org/10.1515/9783110494068}} Wer aber die Diskussion
mit den großen und mittelgroßen Verlagen über deren
Subskriptionsportfolio nicht führt, der hat bestenfalls ein Drittel des
Gesamtmarktes im Blick.

%autor
\begin{center}\rule{0.5\linewidth}{\linethickness}\end{center}

\textbf{Bernhard Mittermaier}, Dr.~rer. nat, hat an der Universität Ulm
in Analytischer Chemie promoviert und an der Humboldt-Universität zu
Berlin den Master of Arts (Library and Information Science) erworben. Er
leitet die Zentralbibliothek des Forschungszentrums Jülich und ist
Mitglied der DEAL-Projektgruppe. ORCiD:
\href{https://orcid.org/0000-0002-3412-6168}{https://orcid.org/0000-0002-3412-6168}

\textbf{Christoph Holzke}, Dr.~rer. nat., promovierte an der Universität
zu Köln im Bereich Pflanzenphysiologie und ist Leiter des Fachbereichs
Wissenschaftliche Dienste in der Zentralbibliothek des
Forschungszentrums Jülich. ORCiD:
\href{https://orcid.org/0000-0002-4937-640X}{https://orcid.org/0000-0002-4937-640X}

\textbf{Claudia Frick}, Dr.~sc. ETH Zürich, promovierte an der ETH
Zürich im Bereich Atmosphärendynamik, studiert derzeit Bibliotheks- und
Informationswissenschaft (MALIS) an der TH Köln und ist Leiterin des
Fachbereichs Literaturerwerbung (Schwerpunkt Wissenschaftliches
Publizieren) in der Zentralbibliothek des Forschungszentrums Jülich.
ORCiD:
\href{https://orcid.org/0000-0002-5291-4301}{https://orcid.org/0000-0002-5291-4301}

\textbf{Irene Barbers} hat wissenschaftliches Bibliothekswesen
(Dipl.-Bibl.) und Bibliotheks- und Informationswissenschaft (MALIS) an
der FH Köln studiert und ist Leiterin des Fachbereichs
Literaturerwerbung (Schwerpunkt Lizenzmanagement) in der
Zentralbibliothek des Forschungszentrums Jülich. ORCiD:
\href{https://orcid.org/0000-0003-2011-7444}{https://orcid.org/0000-0003-2011-7444}

\end{document}
