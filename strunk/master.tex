\documentclass[a4paper,
fontsize=11pt,
%headings=small,
oneside,
numbers=noperiodatend,
parskip=half-,
bibliography=totoc,
final
]{scrartcl}

\usepackage{synttree}
\usepackage{graphicx}
\setkeys{Gin}{width=.4\textwidth} %default pics size

\graphicspath{{./plots/}}
\usepackage[ngerman]{babel}
\usepackage[T1]{fontenc}
%\usepackage{amsmath}
\usepackage[utf8x]{inputenc}
\usepackage [hyphens]{url}
\usepackage{booktabs} 
\usepackage[left=2.4cm,right=2.4cm,top=2.3cm,bottom=2cm,includeheadfoot]{geometry}
\usepackage{eurosym}
\usepackage{multirow}
\usepackage[ngerman]{varioref}
\setcapindent{1em}
\renewcommand{\labelitemi}{--}
\usepackage{paralist}
\usepackage{pdfpages}
\usepackage{lscape}
\usepackage{float}
\usepackage{acronym}
\usepackage{eurosym}
\usepackage[babel]{csquotes}
\usepackage{longtable,lscape}
\usepackage{mathpazo}
\usepackage[normalem]{ulem} %emphasize weiterhin kursiv
\usepackage[flushmargin,ragged]{footmisc} % left align footnote
\usepackage{ccicons} 

%%%% fancy LIBREAS URL color 
\usepackage{xcolor}
\definecolor{libreas}{RGB}{112,0,0}

\usepackage{listings}

\urlstyle{same}  % don't use monospace font for urls

\usepackage[fleqn]{amsmath}

%adjust fontsize for part

\usepackage{sectsty}
\partfont{\large}

%Das BibTeX-Zeichen mit \BibTeX setzen:
\def\symbol#1{\char #1\relax}
\def\bsl{{\tt\symbol{'134}}}
\def\BibTeX{{\rm B\kern-.05em{\sc i\kern-.025em b}\kern-.08em
    T\kern-.1667em\lower.7ex\hbox{E}\kern-.125emX}}

\usepackage{fancyhdr}
\fancyhf{}
\pagestyle{fancyplain}
\fancyhead[R]{\thepage}

% make sure bookmarks are created eventough sections are not numbered!
% uncommend if sections are numbered (bookmarks created by default)
\makeatletter
\renewcommand\@seccntformat[1]{}
\makeatother


\usepackage{hyperxmp}
\usepackage[colorlinks, linkcolor=black,citecolor=black, urlcolor=libreas,
breaklinks= true,bookmarks=true,bookmarksopen=true]{hyperref}
%URLs hart brechen
\makeatletter 
\g@addto@macro\UrlBreaks{ 
  \do\a\do\b\do\c\do\d\do\e\do\f\do\g\do\h\do\i\do\j 
  \do\k\do\l\do\m\do\n\do\o\do\p\do\q\do\r\do\s\do\t 
  \do\u\do\v\do\w\do\x\do\y\do\z\do\&\do\1\do\2\do\3 
  \do\4\do\5\do\6\do\7\do\8\do\9\do\0} 
% \def\do@url@hyp{\do\-} 
\makeatother 

%meta
%meta

\fancyhead[L]{P. Strunk \\ %author
LIBREAS. Library Ideas, 33 (2018). % journal, issue, volume.
\href{http://nbn-resolving.de/}
{}} % urn 
% recommended use
%\href{http://nbn-resolving.de/}{\color{black}{urn:nbn:de...}}
\fancyhead[R]{\thepage} %page number
\fancyfoot[L] {\ccLogo \ccAttribution\ \href{https://creativecommons.org/licenses/by/3.0/}{\color{black}Creative Commons BY 3.0}}  %licence
\fancyfoot[R] {ISSN: 1860-7950}

\title{\LARGE{Personalentwicklung vor Ort}} % title
\author{Petra Strunk} % author

\setcounter{page}{1}

\hypersetup{%
      pdftitle={Personalentwicklung vor Ort},
      pdfauthor={Petra Strunk},
      pdfcopyright={CC BY 3.0 Unported},
      pdfsubject={LIBREAS. Library Ideas, 33 (2018).},
      pdfkeywords={Personalentwicklung, TU9, Universitätsbibliothek, TU Berlin, Austausch, BibHOP, Weiterbildung},
      pdflicenseurl={https://creativecommons.org/licenses/by/3.0/},
      pdfcontacturl={http://libreas.eu},
      baseurl={http://libreas.eu},
      pdflang={de},
      pdfmetalang={de}
     }



\date{}
\begin{document}

\maketitle
\thispagestyle{fancyplain} 

%abstracts

%body
Ortsangabe: Aachen, Berlin, Braunschweig, Darmstadt, Dresden, Hannover,
Karlsruhe, München, Stuttgart.

Ortszeit: 13 Uhr. Los geht's: Acht Bibliotheksmitarbeiter*innen sind zu
Gast bei der neunten. Sie haben den Vormittag mit ihrer Anreise aus
unterschiedlich weit entfernten Städten verbracht und werden nun mit
belegten Brötchen, Obst und alkoholfreien Getränken begrüßt. Sie kennen
sich nicht, kommen aber dennoch sehr schnell ins Gespräch miteinander:
über die Anfahrt, das Wetter, den Anlass ihres Zusammentreffens. In
Berlin ist es eine bunt gemischte Frauengruppe aus Fachangestellten für
Medien- und Informationsdienste, Bibliothekarinnen und einer
Fachreferentin. Die ersten 45 Minuten nach der Ankunft dienen dem
gegenseitigen Kennenlernen. Jede erzählt kurz, seit wann sie im
Bibliothekswesen tätig ist, welche Aufgaben sie aktuell hat und wieso
sie hier gerade vor Ort ist. In den Bibliotheken, aus denen sie kommen,
arbeiten sie in der Ausleihe, im Team Bestandsentwicklung und Metadaten,
im Bereich Elektronisches Publizieren, in der Zeitschriftenbearbeitung,
im Auskunftsdienst, in der Medienbearbeitung, in der Bibliothekstechnik
und im IT-Service. Die Vorstellungsrunde wird kurz unterbrochen, als
Jürgen Christof, der Direktor der Universitätsbibliothek der Technischen
Universität Berlin, herein schaut und die Gäste herzlich begrüßt. Trotz
schlechter Lichtverhältnisse wird schnell ein Gruppenbild mit Herr
erstellt. Wenig später folgt eine Führung durch die Bibliothek, bei der
immer wieder Smartphones gezückt werden, um Eindrücke für die
Kolleg*innen in der Heimatbibliothek festzuhalten. Selbstverbucher und
Rückgabeautomat, Workbay und Carrel, Bücherbox und Doktorandenwagen
werden genauso abgelichtet wie Hinweisschilder, Sitzgelegenheiten und
Informationstheken. Danach beginnen die Hospitationen. Jede Teilnehmerin
konnte schon Wochen im Voraus angeben, welche Arbeitsgebiete sie
besonders interessieren, in welches Team sie hineinschnuppern möchte und
auf welche Fragen sie Antworten sucht. Allein, zu zweit oder zu dritt
informieren sie sich nun über die Open-Access-Aktivitäten der TU Berlin,
die Teaching Library, das Fachreferat Informatik, die Bearbeitung von
Periodika und E-Ressourcen sowie RDA in der Praxis. Sie gehen in die
Büros und an die Schreibtische der Spezialist*innen zu diesen Themen.
Die Fachleute vor Ort zeigen und erläutern ihre Arbeitsprozesse,
informieren über beachtenswerte Aspekte ihrer Aufgaben und beantworten
Fragen der Gäste. Überraschend schnell vermischen sich Geben und Nehmen,
denn die Gäste berichten von ihren eigenen Bemühungen und Erfahrungen,
so dass es für beide Seiten zu einem intensiven Austausch von Hinweisen
und Tipps kommt. Der mit vielfältigen Eindrücken ausgefüllte Tag endet
in einem italienischen Restaurant. Bei Pasta und Prosecco vermischt sich
Bibliothekarisches mit Berlintouristischem.

Die „tatsächliche Bibliotheksarbeit``, die bis hierhin beschrieben
wurde, heißt „BibHOP TU9-Austauschprogramm``. Seit 2006 gibt es den
Zusammenschluss der TU9\footnote{\href{http://www.tu9.de/}{\emph{http://www.tu9.de/}}}:
Neun technische Universitäten haben eine gemeinsame Geschäftsstelle
eingerichtet, um ihre Presse- und Öffentlichkeitsarbeit zu koordinieren,
um Kontakte zu Politik, Wirtschaft und Gesellschaft zu pflegen sowie ein
internationales Marketing zu betreiben. Seit 2014 treffen sich auch die
Bibliotheksleitungen dieser TU9-Universitäten. Sie prüfen, welche Pläne
und Projekte sich für eine Zusammenarbeit eignen. Im Rahmen dieser
Treffen ist unter anderem ein gemeinsames Personalentwicklungsprogramm
konzipiert worden: das BibHOP TU9-Austauschprogramm. Im Oktober 2016
trafen sich die Fort- und Weiterbildungsbeauftragten der neun
Universitätsbibliotheken in Berlin und entwickelten in zwei halben Tagen
dieses Angebot. Sie benannten die Ziele, schrieben die fachliche
Ausrichtung der Hospitationen fest, einigten sich auf den Namen BibHOP
und vereinbarten den genauen Ablauf von Planung und Durchführung eines
möglichen Austauschprogramms. Ihre Arbeitsergebnisse übergaben sie als
Beschlussvorlage den Bibliotheksleitungen, die bereits im Folgemonat
zustimmten, so dass BibHOP 2017 an den Start gehen konnte.

Das Hospitationsprogramm dient dem fachlichen Erfahrungs-und
Informationsaustausch. Es bietet die Möglichkeit, über den Tellerrand zu
schauen, sich zu vernetzen und sich gegenseitig zu inspirieren. Der
kollegiale Austausch begünstigt den Perspektivwechsel und stärkt in
mehrfacher Hinsicht das Zusammengehörigkeitsgefühl: \emph{Wir als
TU9-Bibliotheken, Wir als Bibliothekspersonal, Wir als Mitglieder des
Teams XYZ} und so weiter. Über das gegenseitige Kennenlernen wird ein
\enquote{kurzer Draht} für Fragen, kollegiale Unterstützung und
gemeinsame Projekte geschaffen. Pro Jahr werden insgesamt 72 Plätze
angeboten: Jede Bibliothek betreut acht Personen und kann acht Personen
in die anderen Bibliotheken entsenden. Das bedeutet, dass jede*r allein
unterwegs ist, vor Ort aber neue Kolleg*innen kennenlernt. Ein
Aufenthalt dauert inklusive Hin- und Rückfahrt zwei Tage. Vor Ort können
sich die Kolleg*innen bei einer Hausführung über die Dienstleistungen
der besuchten Bibliothek informieren und zudem in zuvor ausgewählten
Arbeitsbereichen hospitieren. Die mittlerweile so genannten BibHopper
oder auch BibHopser müssen sich bewerben und angeben, welche Bibliothek
sie besuchen möchten und was genau sie dort interessiert. Dafür stellt
jede TU9-Bibliothek ein Portfolio zusammen. Bei der Auswahl werden vor
allem Mitarbeiter*innen berücksichtigt, die sonst wenig oder keine
Gelegenheit zu Dienstreisen haben. Kosten und Aufwand werden so gering
wie möglich gehalten. Wir rechnen für die notwendigen Dienstreisen mit
insgesamt nicht mehr als 2.640 Euro je Bibliothek pro Jahr. Neben diesen
Kosten müssen die durch die Hospitation gebundenen Personalressourcen
sowie der Arbeitsaufwand der betreuenden Mitarbeiter*innen beachtet
werden.

Nicht unerwähnt bleiben soll die konstruktive und unkomplizierte
Zusammenarbeit der Fort- und Weiterbildungsbeauftragten in den
TU9-Bibliotheken. Gemeinsam haben wir Bewerbungsformular, Feedbackbogen
und Teilnahmebestätigung entwickelt. Wir nutzen als Arbeitsplattform das
Confluence-Projects-Wiki der TIB Hannover und als Kommunikationsmittel
eine Mailingliste, die in München eingerichtet wurde. Alle
Teilnehmer*innen waren nach ihren Aufenthalten begeistert über die
offene Atmosphäre, in der Einblicke in Arbeitsabläufe gewährt wurde, die
so nicht auf Tagungen oder bei anderen offiziellen Zusammenkünften
vermittelt werden können. Fotos und Berichte wurden in den eigenen Teams
präsentiert und Vergleiche zur eigenen Situation gezogen. Das Programm
ist in allen TU9-Bibliotheken gleichermaßen gut angenommen worden und
wird deshalb ein zweites Mal durchgeführt. Ende 2018 werden wir beraten,
ob es eine Fortsetzung geben wird. BibHOP -- ein echter Ortstermin!

%autor
\begin{center}\rule{0.5\linewidth}{\linethickness}\end{center}

\textbf{Petra Strunk} leitet das Querschnittsreferat Personalentwicklung
/ Ausbildung der Universitätsbibliothek der Technischen Universität
Berlin.

\end{document}
